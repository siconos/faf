\documentclass[8pt,red]{beamer}
\usepackage{amsmath,amsthm,amsfonts,amssymb,amsbsy}
\everymath{\displaystyle}
\usepackage{stmaryrd} 
\usepackage{subeqnarray}

\usepackage{cases}
\usepackage{array}
\usepackage{pifont}

\usepackage{fancyhdr}

\makeatletter
  \def\Hy@PageAnchorSlidesPlain{}%
  \def\Hy@PageAnchorSlide{}%
\makeatother


\usepackage{lastpage}
\usepackage{subfig}
\usepackage{float}

\usepackage{graphicx}
\usepackage{subfloat}
\usepackage{rotating}
\usepackage{tikz}

\usetikzlibrary{arrows}
\usetikzlibrary{calc}


% \usepackage{algorithm}
% \usepackage{algorithmic}


\usepackage{minted}
\usepackage{color}



%% Symbole de fraction
\newcommand{\Frac}[2]{{\displaystyle \frac{\displaystyle #1}{\displaystyle #2}}}
\newcommand{\Prac}[2]{\displaystyle \genfrac{(}{)}{}{}{\displaystyle #1}{\displaystyle #2}}
\newcommand{\Crac}[2]{\displaystyle \genfrac{[}{]}{}{}{\displaystyle #1}{\displaystyle #2}}

\newcommand{\norme}[1]{\|#1\|}


\newcommand{\HRule}{\rule{\linewidth}{1mm}}

% Fonction mathematiques

\newcommand{\transposee}[1]{{\vphantom{#1}}^{\text{\tiny{\textsf T}}}{#1}}
\newcommand{\argmin}{\mathop{\mathrm{argmin}}}
\newcommand{\argminn}{\mathop{\mathrm{argmin}}}
\newcommand{\lexicomin}{\mathop{\mathrm{lexicomin}}}
%\newcommand{\arg}{\mathop{\mathrm{arg}}}



\DeclareMathOperator{\rot}{rot}
\DeclareMathOperator{\sh}{sh}
\DeclareMathOperator{\ch}{ch}
%\DeclareMathOperator{\th}{th}
\DeclareMathOperator{\arcsh}{arcsh}
\DeclareMathOperator{\argth}{argth}
\DeclareMathOperator{\sign}{sign}


%%The Principal Value Integral symbol
\def\Xint#1{\mathchoice
   {\XXint\displaystyle\textstyle{#1}}%
   {\XXint\textstyle\scriptstyle{#1}}%
   {\XXint\scriptstyle\scriptscriptstyle{#1}}%
   {\XXint\scriptscriptstyle\scriptscriptstyle{#1}}%
   \!\int}
\def\XXint#1#2#3{{\setbox0=\hbox{$#1{#2#3}{\int}$}
     \vcenter{\hbox{$#2#3$}}\kern-.5\wd0}}
\def\ddashint{\Xint=}
\def\dashint{\Xint-}



% macro pour les symbols d'ensemble
%\nbOne
\def\nbOne{{\mathchoice{\rm 1\mskip-4mu l}{\rm 1\mskip-4mu l} {\rm 1 \mskip-4.5mu l}{\rm 1\mskip-5mu l}}}
%
%%  Les ensembles de nombres  C. Fiorio (fiorio@math.tu-berlin.de) 
%
\def\nbR{\ensuremath{\mathrm{I\!R}}} % IR
\def\nbN{\ensuremath{\mathrm{I\!N}}} % IN
\def\nbF{\ensuremath{\mathrm{I\!F}}} % IF
\def\nbH{\ensuremath{\mathrm{I\!H}}} % IH
\def\nbK{\ensuremath{\mathrm{I\!K}}} % IK
\def\nbL{\ensuremath{\mathrm{I\!L}}} % IL
\def\nbM{\ensuremath{\mathrm{I\!M}}} % IM
\def\nbP{\ensuremath{\mathrm{I\!P}}} % IP
%
% \nbOne : 1I : symbol one
\def\nbOne{{\mathchoice {\rm 1\mskip-4mu l} {\rm 1\mskip-4mu l}
{\rm 1\mskip-4.5mu l} {\rm 1\mskip-5mu l}}}
%
% \nbC   :  Nombres Complexes
\def\nbC{{\mathchoice {\setbox0=\hbox{$\displaystyle\rm C$}%
\hbox{\hbox to0pt{\kern0.4\wd0\vrule height0.9\ht0\hss}\box0}}
{\setbox0=\hbox{$\textstyle\rm C$}\hbox{\hbox
to0pt{\kern0.4\wd0\vrule height0.9\ht0\hss}\box0}}
{\setbox0=\hbox{$\scriptstyle\rm C$}\hbox{\hbox
to0pt{\kern0.4\wd0\vrule height0.9\ht0\hss}\box0}}
{\setbox0=\hbox{$\scriptscriptstyle\rm C$}\hbox{\hbox
to0pt{\kern0.4\wd0\vrule height0.9\ht0\hss}\box0}}}}
%
% \nbQ   : Nombres Rationnels Q
\def\nbQ{{\mathchoice {\setbox0=\hbox{$\displaystyle\rm
Q$}\hbox{\raise
0.15\ht0\hbox to0pt{\kern0.4\wd0\vrule height0.8\ht0\hss}\box0}}
{\setbox0=\hbox{$\textstyle\rm Q$}\hbox{\raise
0.15\ht0\hbox to0pt{\kern0.4\wd0\vrule height0.8\ht0\hss}\box0}}
{\setbox0=\hbox{$\scriptstyle\rm Q$}\hbox{\raise
0.15\ht0\hbox to0pt{\kern0.4\wd0\vrule height0.7\ht0\hss}\box0}}
{\setbox0=\hbox{$\scriptscriptstyle\rm Q$}\hbox{\raise
0.15\ht0\hbox to0pt{\kern0.4\wd0\vrule height0.7\ht0\hss}\box0}}}}
%
% \nbT   : T
\def\nbT{{\mathchoice {\setbox0=\hbox{$\displaystyle\rm
T$}\hbox{\hbox to0pt{\kern0.3\wd0\vrule height0.9\ht0\hss}\box0}}
{\setbox0=\hbox{$\textstyle\rm T$}\hbox{\hbox
to0pt{\kern0.3\wd0\vrule height0.9\ht0\hss}\box0}}
{\setbox0=\hbox{$\scriptstyle\rm T$}\hbox{\hbox
to0pt{\kern0.3\wd0\vrule height0.9\ht0\hss}\box0}}
{\setbox0=\hbox{$\scriptscriptstyle\rm T$}\hbox{\hbox
to0pt{\kern0.3\wd0\vrule height0.9\ht0\hss}\box0}}}}
%
% \nbS   : S
\def\nbS{{\mathchoice
{\setbox0=\hbox{$\displaystyle     \rm S$}\hbox{\raise0.5\ht0%
\hbox to0pt{\kern0.35\wd0\vrule height0.45\ht0\hss}\hbox
to0pt{\kern0.55\wd0\vrule height0.5\ht0\hss}\box0}}
{\setbox0=\hbox{$\textstyle        \rm S$}\hbox{\raise0.5\ht0%
\hbox to0pt{\kern0.35\wd0\vrule height0.45\ht0\hss}\hbox
to0pt{\kern0.55\wd0\vrule height0.5\ht0\hss}\box0}}
{\setbox0=\hbox{$\scriptstyle      \rm S$}\hbox{\raise0.5\ht0%
\hboxto0pt{\kern0.35\wd0\vrule height0.45\ht0\hss}\raise0.05\ht0%
\hbox to0pt{\kern0.5\wd0\vrule height0.45\ht0\hss}\box0}}
{\setbox0=\hbox{$\scriptscriptstyle\rm S$}\hbox{\raise0.5\ht0%
\hboxto0pt{\kern0.4\wd0\vrule height0.45\ht0\hss}\raise0.05\ht0%
\hbox to0pt{\kern0.55\wd0\vrule height0.45\ht0\hss}\box0}}}}
%
% \nbZ   : Entiers Relatifs Z
\def\nbZ{{\mathchoice {\hbox{$\sf\textstyle Z\kern-0.4em Z$}}
{\hbox{$\sf\textstyle Z\kern-0.4em Z$}}
{\hbox{$\sf\scriptstyle Z\kern-0.3em Z$}}
{\hbox{$\sf\scriptscriptstyle Z\kern-0.2em Z$}}}}
%%%% fin macro %%%%



\newcommand{\putidx}[1]{\index{#1}\textit{#1}}


%\definecolor{darkgray}{gray}{.25}
\definecolor{gray}{gray}{.5}
\definecolor{lightgray}{gray}{.75}
%\definecolor{gradbegin}{rgb}{0,1,1}
%\definecolor{gradend}{rgb}{0,.1,.95}
%\newcommand{\newtexte}[1]{\textcolor{darkgray} {#1}}
\newcommand{\newtexte}[1]{{#1}}% macro pour les varibales favorites
% normal tangent
\def\n{{\hbox{\tiny{N}}}}
\def\t{{\hbox{\tiny{T}}}}
\def\tone{{\hbox{\tiny{T}}}_{\hbox{\tiny{1}}}}
\def\ttwo{{\hbox{\tiny{T}}}_{\hbox{\tiny{2}}}}
\def\ss{{\hbox{\tiny{S}}}}
\def\nt{\hbox{\tiny{NT}}}
\def\nsf{\hbox{\tiny{\textsf N}}}
\def\tsf{\hbox{\tiny{\textsf T}}}
\def\sigman{\sigma_{\n}}
\def\sigmat{\sigma_{\t}}
\def\sigmant{\sigma_{\nt}}
\def\epsn{\epsilon_{\n}}
\def\epst{\epsilon_{\t}}
\def\epsnt{\epsilon_{\nt}}
\def\eps{\epsilon}
\def\veps{\varepsilon}
\def\sig{\sigma}
\def\Rn{R_{\n}}
\def\Rt{R_{\t}}
\def\cn{c_{\n}}
\def\Cn{C_{\n}}
\def\ct{c_{\t}}
\def\Ct{C_{\t}}
\def\un{u_{\n}}
\def\ut{\buu_{\t}}
\def\uut{u_{\t}}
\def\unc{u_{\n}^c}
\def\utc{\buu_{\t}^c}
\def\vn{v_{\n}}
\def\vt{v_{\t}}
\def\rr{\hbox{\tiny{\textsf R}}}
\def\irr{\hbox{\tiny{\textsf{IR}}}}
\def\rn{r_{\n}}
\def\rt{\brr_{\t}}
\def\rnc{r_{\n}^c}
\def\rtc{\brr_{\t}^c}
\def\trn{\Tilde{r}_{\n}}
\def\trt{\Tilde{\brr}_{\t}}
\def\tr{\Tilde{\brr}}
\def\tv{\Tilde{\bvv}}
\def\vn{v_{\n}}
\def\vt{\bvv_{\t}}
\def\adh{\mathsf{adh}}
\def\adj{\hbox{\tiny{\textsf{adj}}}}
\def\adjc{\hbox{\tiny{\textsf{adjC}}}}
\def\adja{\hbox{\tiny{\textsf{adjA}}}}
\def\cc{\hbox{\tiny{\textsf C}}}
\def\ca{\hbox{\tiny{\textsf A}}}

% domaines et frontieres
\def\om{\Omega}
\def\oma{\Omega^{\alpha}}
\def\omu{\Omega^1\cup \Omega^2}
\def\gc{\Gamma_c}
\def\omt{\omu \cup \gc}
% derivee partielle et gradient et divergence
\def\p{\partial}
\def\grad{\nabla}
\def\div{\mathop{\rm div}\nolimits}
%

%\DeclareTextSymbol{\deg}{T1}{6}
%\def\degre{\mathdegree}
%\newcommand{\degre}{\mathdegree}

\def\etc{\textit{etc}\ldots}
\newcommand{\mdegre}{\hbox{\text{\degre}}}

%\def\nscd{\textsf{\bfseries NSCD}}
%\def\nscd{\textsf{NSCD}}
\newcommand{\nscd}{\textsf{NSCD}}
%\Pisymbol{psy}{212} ou encore \Pisymbol{psy}{228}




%----------------------------------------------------------------------
%             Des chiffres avec des ronds autour
%----------------------------------------------------------------------
\def\nombrecercle#1{\def\taille{0.3}
                \put(0,0){#1}
                \put(0.08,0.08){\circle{\taille}}}



\def\ae#1{\stackrel{\mbox{\scriptsize a.e.}}{#1}}
\def\argmin{\mathop{\rm argmin}}
\def\eqref#1{{\rm (\ref{#1})\/}}
\def\indicfon{\mathord{\rm i}}       %indicator function
\def\p{\mathord{\rm proj}}
\def\N{\mathord{\rm N}}
% \def\prosca#1#2{#1\cdot#2}
\def\prosca#1#2{\langle #1,#2\rangle}
\def\qedtext{\mbox{}\hfill$\Box$}
\def\qedmath{\eqno\Box}

\def\s{{$\mathcal{S}$}}
\def\somme{\mathop{\textstyle\sum}}
\def\somme{\mathop{\textstyle\sum}}
\def\submoins{_{\scriptscriptstyle-}}
\def\subplus{_{\scriptscriptstyle+}}
\def\T{\mathord{\rm T}}

%----------------------------------------------------------------------
%             Macro M Jean 
%----------------------------------------------------------------------

\def\Real{\mbox{I\hspace{-.15em}R}}
\def\Integer{\mbox{I\hspace{-.15em}N}}
\def\Bunit{\mbox{I\hspace{-.15em}B}}
\def\real{\mbox{\scriptsize I\hspace{-.15em}R}}
\def\bunit{\mbox{\scriptsize I\hspace{-.15em}B}}
\def\IL{\mbox{\scriptsize I\hspace{-.15em}L}}
\def\Indic{\mbox{\large $\psi$}}
\def\bfxi{\mbox{$\xi$ \hspace{-1.1em} $\xi$}}
%\def\bfXi{\mbox{$\Xi$ \hspace{-1.1em} $\Xi$}}
\def\RunR{\mathcal R}
\def\RunRN{\mathcal R_{N}}
\def\RunRT{\mathcal R_{T}}
\def\RunS{\mathcal S}
\def\RunSN{\mathcal S_{N}}
\def\RunST{\mathcal S_{T}}
\def\RunU{\mathcal U}
\def\RunUN{\mathcal U_{N}}
\def\RunUT{\mathcal U_{T}}
\def\RunUP{\mathcal U'}
\def\RunUPN{\mathcal U'_{N}}
\def\RunUPT{\mathcal U'_{T}}
\def\RunJ{\mathcal J}
\def\RunW{\mathcal W}
\def\RunF{f}
\def\RunFa{f_{1}}
\def\RunFb{f_{2}}
\def\RunFP{f'}
\def\RunV{v}
\def\RunVP{v'}
\def\EspF{\mathcal F}
\def\EspV{\mathcal V}

\def\N{\mbox{I\hspace{ -.15em}N}}
\def\Z{\mbox{Z\hspace{ -.3em}Z}}
\def\Q{\mbox{l\hspace{ -.47em}Q}}
\def\R{\mbox{l\hspace{ -.15em}R}}
\def\F{\mbox{l\hspace{ -.15em}F}}
\def\E{\mbox{l\hspace{ -.15em}E}}
\def\LMGC90{{\small \it LMGC90 }}
\def\NSCD{{\small \it NSCD }}
\def\CHIC{{\small \it CHIC }}
\def\half{{\frac{_{1}}{^{2}}}}
\def\12T{{\frac{_{1}}{^{2T}}}}

\def\geq{\geqslant}
\def\leq{\leqslant}
\def\ge{\geqslant}
\def\le{\leqslant}


\begingroup
\count0=\time \divide\count0by60 % Hour
\count2=\count0 \multiply\count2by-60 \advance\count2by\time
% Min
\def\2#1{\ifnum#1<10 0\fi\the#1}
\xdef\isodayandtime{\the\year-\2\month-\2\day\space\2{\count0}:%
\2{\count2}}
\endgroup

%---------------------------------------------------------------------
%             Redaction note environnement B. Brogliato
%----------------------------------------------------------------------
\makeatletter

{\newtheorem{ndr1bb}{\textbf{\textsc{Redaction note B.B.}}}[section]}

\newenvironment{ndrbb}%
{%
\noindent\begin{ndr1bb}\hrule\vspace{1em}%
\ttfamily\small
}%
{%
\begin{flushright}%
%\vspace{-1.5em}\ding{111}
\end{flushright}%
\vspace{-1.5em}\hrule
\end{ndr1bb}%
}
%----------------------------------------------------------------------
%             Redaction note environnement V.ACARY
%----------------------------------------------------------------------
% Faut etre fou pour s'amuser a pondre des notes pareilles

{\newtheorem{ndr1va}{\textbf{\textsc{\footnotesize Redaction note V.A.}}}[section]}

\newenvironment{ndrva}%
{%
\noindent\begin{ndr1va}\hrule\vspace{1em}%
\ttfamily\small \  \\
\noindent}%
{%
$ $ \\
\hrule
\end{ndr1va}%
}
\makeatother








% ----------------DEFINITIONS-----------------
% 

 \def\II{\mathop{{\rm I}\mskip-3.0mu{\rm I}}\nolimits}




% -----------------------------------
 \def\c{\mathop{{\rm 1}\mskip-10.0mu{\rm C}}\nolimits}
 \def\C{\mathop{{\rm 1}\mskip-10.0mu{\rm C}}\nolimits}
 \def\ZZ{\mathaccent23Z}
% 

\newcommand{\ie}{{\textit{i.e.}}}


%\def\sgn{\mbox{\rm sgn}}
\DeclareMathOperator{\sgn}{sgn}
\DeclareMathOperator{\proj}{proj}
\DeclareMathOperator{\prox}{prox}
\DeclareMathOperator{\co}{co}


%\newcommand{\RR}{\mbox{\rm $I\!\!R$}}
%\newcommand{\NN}{\mbox{\rm $I\!\!N$}}


\def\RR{\nbR}
\def\NN{\nbN}

% ---------------- MMC -----------------
% 

\newcommand{\contract}{{\,:\,}}

\newcommand{\scontract}{{\,{\Bar\otimes}\,}}
\newcommand{\tcontract}{{\,{\Bar{\Bar{\Bar\otimes}}}\,}}


\newcommand{\DP}[2]{\displaystyle \frac{\partial {#1}}{\partial {#2}}}

\newtheorem{definition}{Definition}
\newtheorem{proposition}{Proposition}
\newtheorem{lemma}{Lemma}

\newtheorem{claim}{Claim}
\newtheorem{remark}{Remark}
\newtheorem{assumption}{Assumption}
\newtheorem{example}{Example}
\newtheorem{conjecture}{Conjecture}
\newtheorem{corollary}{Corollary}
\newtheorem{OP}{OP}
\newtheorem{problem}{Problem}
\newtheorem{theorem}{Theorem}


\def\dt{{\rm d}t}
\def\dv{{\rm d}v}
\def\di{{\rm d}i}
\def\dI{{\rm d}I}
\def\dU{{\rm d}U}


\def\nat{{\hbox{\sf \tiny{nat}}}}
\def\nor{{\hbox{\sf \tiny{nor}}}}
\def\fb{\hbox{\tiny{\textsf FB}}}
\def\vione{{\hbox{\tiny{vi-1}}}}
\def\vitwo{{\hbox{\tiny{vi-2}}}}
\def\qvitwo{{\hbox{\tiny{qvi-2}}}}
\def\mjone{{\hbox{\tiny{mj-1}}}}
\def\mjtwo{{\hbox{\tiny{mj-2}}}}
\def\acone{{\hbox{\tiny{ac-1}}}}
\def\actwo{{\hbox{\tiny{ac-2}}}}





\begingroup
\count0=\time \divide\count0by60 % Hour
\count2=\count0 \multiply\count2by-60 \advance\count2by\time
% Min
\def\2#1{\ifnum#1<10 0\fi\the#1}
\xdef\isodayandtime{\the\year-\2\month-\2\day\space\2{\count0}:%
\2{\count2}}
\endgroup
%%%% fin macro %%%%




\usepackage{wasysym}

\makeatletter
\newcommand{\pushright}[1]{\ifmeasuring@#1\else\omit\hfill$\displaystyle#1$\fi\ignorespaces}
\newcommand{\pushleft}[1]{\ifmeasuring@#1\else\omit$\displaystyle#1$\hfill\fi\ignorespaces}
\makeatother


\setbeamertemplate{theorem}[ams style]
\setbeamertemplate{theorems}[numbered]
\usepackage[backend=biber,style=authoryear,bibstyle=authoryear,citestyle=authoryear-comp,natbib=true,maxcitenames=2,uniquelist=false,uniquename=false,bibencoding=utf8]{biblatex}
\addbibresource{./biblio/String.bib}
\addbibresource{./biblio/NonSmooth.bib}
\addbibresource{./biblio/Math.bib}
\addbibresource{./biblio/Multibody.bib}
\addbibresource{./biblio/Fem.bib}
\addbibresource{./biblio/Dae.bib}
\addbibresource{./biblio/Meca.bib}
\addbibresource{./biblio/AnaNum.bib}
\addbibresource{./biblio/Math-Impact.bib}
\addbibresource{./biblio/Contact.bib}
\addbibresource{./biblio/Optim.bib}
\addbibresource{./biblio/Cp.bib}

% %hideothersubsections
% \usetheme[hideothersubsections,width=2.5cm]{Goettingen}
% %\useoutertheme[headline=empty]{miniframes}

\usetheme[]{Montpellier}
\makeatletter
\useoutertheme[height=0pt, width=0cm]{sidebar}
{\usebeamercolor{structure}}
\setbeamertemplate{sidebar canvas \beamer@sidebarside}[vertical shading][top=structure.fg!25,bottom=structure.fg!10]

%%% Local Variables: 
%%% mode: latex
%%% TeX-master: t
%%% End: 

\setcounter{tocdepth}{1}
% \title{Formulations and extensive comparisons of 3D frictional contact solvers based on performance profiles}
% \author{Vincent Acary, Maurice Br\'emond, Olivier Huber \\ INRIA Rh\^one--Alpes, Grenoble.}
% \date{CMIS 2018, Biella, Italy}


\title[Coulomb friction and optimisation ]{Numerical solutions of the Coulomb friction contact problem\\ from the perspective of\\  optimisation and mathematical programming\\[5mm]
  \large{ICCCM, Munich,   July 2025}}

\author{Vincent Acary}
\date{
   \includegraphics[height=0.15\textheight]{./logos/inr_logo_rouge.jpg}\hfill
   \includegraphics[height=0.15\textheight]{./logos/logo_ljk2.pdf}\hfill
   \includegraphics[height=0.15\textheight]{./logos/logo_uga_transparent.png}\\
  
 }
 \institute{Inria -  Centre de l'Université Grenoble Alpes - Laboratoire Jean Kuntzmann}


\setbeamertemplate 
{footline} 
{\quad\hfill\strut\insertsection\quad--\quad\insertframenumber/\inserttotalframenumber\strut\quad\quad} 

\graphicspath{{../figure/}}

% \includeonly{%
% %  introduction,
%   fc3d,
%   existence,
%   numerics,
%   comparison
% }



%\newtheorem{defn}{Definition}
%\renewcommand{\thedefn}{\arabic{defn}}
% \newtheorem{thm}[defn]{Theorem}
% \newtheorem{corr}[defn]{Corollary}
% \newtheorem{ass}[defn]{Assumption}
% \newtheorem{lem}[defn]{Lemma}
% \newtheorem{rem}[defn]{Remark}
% \newtheorem{hypo}[defn]{Hypotheses}
% \newtheorem{exmp}[defn]{Example}
% \newtheorem{prop}[defn]{Proposition}
\newcommand{\diag}{\mbox{\rm diag}}
\newcommand{\co}{\overline{\mathit{co}}}
\newcommand{\rect}{\overline{\mathit{rect}}}
\newcommand{\newb}{g}

\renewcommand{\tr}[1]{\textcolor{red}{#1}}


\newcommand{\norm}[1]{\lVert#1\rVert}

\usepackage{hyperref}
\hypersetup{
  colorlinks,
  citecolor=violet,
  linkcolor=red,
  urlcolor=blue}

\begin{document}


\frame{\titlepage
  \thispagestyle{empty}\addtocounter{framenumber}{-1}
}


%\section{Introduction}

% \begin{frame}
% %  \frametitle{Introduction. Bio.}
%   \begin{block}
%     {Team-Project TRIPOP.\\ INRIA. Centre de recherche de l'Université Grenoble Alpes (UGA)}
%     ``Jean Jacques Moreau's fan club''. Convex Analysis and Nonsmooth Mechanics.
%     \begin{itemize}
%     %\item Scientific leader :  Bernard Brogliato
%     \item $6$ permanents, $6$ PhD, $4$ Post-docs, $3$ Engineer,
%     \item Nonsmooth simulation and numerical modeling for natural gravitational risk in mountains.
%     \item Nonsmooth dynamical systems :
%       Modeling, analysis, simulation and Control.
%     \end{itemize}
%   \end{block}
%     \begin{block}
%       {Current Personal research themes}
%     \begin{itemize}
%     \item Nonsmooth Dynamical systems in the large:\\
%       Higher order Moreau's sweeping process. Complementarity systems and Filippov systems
%     \item Time--integration techniques for nonsmooth mechanical systems:\\ Mixed higher order schemes, Time--discontinuous Galerkin methods, Projected time--stepping schemes and generalized $\alpha$--schemes.
%     %\item Modeling and simulation of switched electrical circuits
%     %\item Discretization method for sliding mode control and Optimal control.
%     \item Formulation and numerical solvers for Coulomb's friction and Signorini's problem.% Second order cone programming.
%     \item Non-associated plasticity of geomaterials with contact, friction and impact
%     \item Coupling SPH/DEM/FEM and MPM/DEM/FEM
%     \item Data-driven modeling and data assimilation.
%     \end{itemize}
%   \end{block}
% \end{frame}

% \begin{frame}
%   \frametitle{Motivations}
%   TODO

  
%   \tr{Beyond} the numerical simulation of frictional contact problems (Signorini + friction)
%   \begin{itemize}
%   \item Few mathematical results: existence, uniqueness, convergence, rate of convergence.
%   \item Need for comparisons on a fair basis: implementation (software) and benchmarks (data)
%   \item Without convergence proof, test your new  method on a large set of  benchmarks shared by the community. (a common practice in numerical optimization).
%   \item Open and reproducible science.
%   \end{itemize}
% \end{frame}

\begin{frame}
  \frametitle{Motivations \&  contents}

  \begin{enumerate}
  \item Introduce a sufficiently generic and representative discrete 3D frictional contact problem.
  \item Interpret this problem in the context of numerical optimisation and mathematical programming.
  \item Provide an existence result, whose assumption can be verified numerically.
  \item Compare the main existing numerical methods based on a large collection of problems (FCLIB) and a common implementation (SICONOS/Numerics).
  \item Propose a new solution method based on the interior point method.
  \end{enumerate}
  \addtocounter{page}{-2}
\end{frame}

\frame{\thispagestyle{empty}\tableofcontents\addtocounter{framenumber}{-1}}


%\section{Introduction}
\label{Sec:Introduction}

\frame{
  \frametitle{Contact and Friction in Mechanics}
}



%%% Local Variables: 
%%% mode: latex
%%% TeX-master: "s"
%%% End: 

%\frame{\tableofcontents}


\section{The discrete frictional contact problem}
\label{Sec:fc3d}

\subsection{Signorini condition and Coulomb's friction}

\frame{
  \frametitle{Signorini's condition and Coulomb's friction} 
  \begin{minipage}[c]{0.45\linewidth}
    \begin{tikzpicture}[ scale=2,
      axis/.style={ ->, >=stealth'},
      normal/.style={ thick, ->, >=stealth'},
      important line/.style={very thick}, 
      dashed line/.style={dashed, thin},
      every node/.style={color=black},
      soldot/.style={only marks,mark=*},
      holdot/.style={fill=white,only marks,mark=*}
      ]
      % body
      \node (BodyA) at (1,-1) {Body A};
      \fill[gray!20] (1,0) arc (0:-90:1);
      \fill[gray!20] (1,0) arc (90:180:1);
      \draw (1,0) arc (90:180:1);

      \node (BodyB) at (-1,1) {Body B};
      \draw (0,1) arc (0:-90:1);
      \fill[gray!20] (0,1) arc (90:180:1);
      \fill[gray!20] (0,1) arc (0:-90:1);

      % local frame
      \def\nlength{0.35};
      \coordinate (CA)  at  ({1.0-sqrt(2)/2.0},{-1.0+sqrt(2)/2.0});
      \node[] at  (CA) [right] {$\sf C_A$};
      \draw[holdot]  (CA) circle(0.05em);
      \draw[normal] (CA) -- ($(CA)+({-\nlength*sqrt(2)/2.0},{+\nlength*sqrt(2)/2.0 })$) node [right] {$\,\sf N$};
      \draw[normal] (CA) -- ($(CA)+({-\nlength*sqrt(2)/2.0},{-\nlength*sqrt(2)/2.0 })$) node [above] {$\sf T_1\quad$};
      \draw[dashed line] (BodyA) -- (BodyB);
      \draw[holdot] ($(CA)+({\nlength*sqrt(3)/2.0},{0.0})$) circle(0.2em);
      \node at ($(CA)+({\nlength*sqrt(3)/2.0},{0.0})$) [right]{$\sf T_2$};
      \draw[soldot] ($(CA)+({\nlength*sqrt(3)/2.0},{0.0})$) circle(0.02em);
      
      \coordinate (CB)  at  ({-1.0+sqrt(2)/2.0},{1.0-sqrt(2)/2.0});
      \node at  (CB) [above] {$\sf C_B$};
      \draw[holdot]  (CB) circle(0.05em);

      \draw[axis] (CA) -- (CB) node[midway, below left ] {$\sf g_\n$} ;

      % \draw[axis] (0,-0.4) -- (0,0.4) node(yline)[right] {$\sgn(x)$};
      % % lines
      % \draw[important line] (-0.4,-0.3) -- (0.   ,-.3);
      % \draw[important line] (0.0,0.3) --(.4,.3)  ;
      % \coordinate (O) at (0.0, 0.05);
      % \draw[fill] (O) circle (0.03em);
      % \draw (0.0,0.05) node[right]{$a$};
      % \draw (0.0,0.3) node[left]{$1$};
      % \draw (0.0,-0.3) node[right]{$-1$};
      % \draw[holdot] (0.0,0.3) circle (0.03em);
      % \draw[holdot] (0.0,-.3) circle (0.03em);
    \end{tikzpicture}
  \end{minipage}
\begin{minipage}[c]{0.49\linewidth}
    \begin{itemize}
    \item gap function $ g_{\n} = (C_B-C_A) \sf N.$
    \item reaction forces and  velocities
      $$r =  r_\n {\sf N} + r_\t, \quad \text{ with  } r_\n \in \nbR, \quad r_\t \in \nbR^2.$$
      $$u =  u_\n {\sf N} + u_\t, \quad \text{ with } u_\n \in \nbR \quad u_\t \in \nbR^2.$$
    \item Signorini conditions
      $$  \text{ position level : }  0 \leq g_{\n} \perp r_\n \geq 0.$$\\
      $$ \text{velocity level : }\left\{\begin{array}{ll}
            0 \leq u_{\n} \perp r_\n \geq 0  &\text{ if } g_{\n} \leq 0\\
            r_{\n} =0 &\text{ otherwise}.
          \end{array}\right.$$
    \end{itemize}

  \end{minipage}
}
\frame{
  \frametitle{Signorini's condition and Coulomb's friction} 
  \begin{block}{Coulomb friction modeling assumption}
   Let $\mu$ be the coefficient of friction.  Let us define the Coulomb friction cone $K$ which is chosen as the
    isotropic second order cone
    \begin{equation}
      \label{eq:CoulombCone}
      K = \{r \in \nbR^3 \mid \|r_\t\| \leq \mu r_\n\}.
    \end{equation}
    Coulomb friction postulates
    \begin{itemize}
    \item for the \tr{sticking case} that
      \begin{equation}
        \label{eq:Coulom-stick}
        u_{\t} =0,\quad r \in K,
      \end{equation}
    \item and for the \tr{sliding case} that
      \begin{equation}
        \label{eq:Coulom-slide}
        u_{\t}  \neq 0,\quad \|r_\t\| = \mu r_\n , \quad  r_\t = - \frac{u_\t}{\|u_\t\|}\|r_\t\|.
      \end{equation}
    \end{itemize}
  \end{block}

  \begin{block}
    {Disjunctive formulation of the frictional contact behavior}
    \begin{equation}
      \label{eq:contact-disjunctive}
      \left\{\begin{array}{llr}
          r = 0  &\text{ if } g_{\n} > 0  & \text{(no contact)}\\
          r = 0,  u_\n \geq 0   &\text{ if } g_{\n} \leq 0 & \text{(take--off)} \\
          r \in K, u =0 &\text{ if } g_{\n} \leq 0 & \text{(sticking)}  \\
               r \in \partial K,u _\n=0, {r_\t} = - \frac{u_\t}{\|u_\t\|}\|r_\t\|  &\text{ if } g_{\n} \leq 0 & \text{(sliding)}  \\
        \end{array}\right.
    \end{equation}
  \end{block}
  

}

\frame{
  \frametitle{Signorini's condition and Coulomb's friction} 
  
  \begin{block}{Second Order Cone Complementarity (SOCCP) formulation}
  
    \begin{itemize}
    \item Modified relative velocity $\tilde u \in \nbR^3$ \citep{DeSaxce92} defined by 
      \begin{equation}
        \label{eq:modified-velocity}
        \tilde u = u +\mu \|u_\t\| \sf N.
      \end{equation}
    
    \item  Second-Order Cone Complementarity Problem (SOCCP) 
      \begin{equation}
        \label{eq:contact-SOCCP}
        K^\star \ni \tilde u \perp r \in K
      \end{equation}
      if $ g_\n \leq 0 $ and $r=0$ otherwise.\\[1mm]
      The set $ K^\star $ is
      the dual convex cone to $K$ defined by
      \begin{equation}
        \label{eq:dual-cone}
        K^\star = \{u \in \nbR^3 \mid  r^\top u \geq 0, \quad \text{for all } r \in K   \}.
      \end{equation}

    \end{itemize}
 \end{block}
    \vfill \small
    \citep{Acary.Brogliato2008,Acary.ea_ZAMM2011}

}
\frame{
  \frametitle{Signorini's condition and Coulomb's friction} 
   \begin{figure}[htbp]
  \centering
  \resizebox{!}{0.8\textheight}{\input{../figure/cone1-b.pdf_t}}
  \caption{Coulomb's friction and the modified velocity $\tilde u$. The sliding case.}
  \label{fig:CoulombFrictionSlidingDeSaxce}
\end{figure} 
}

\subsection{Definition}


% \frame{
%   \frametitle{3D frictional contact problem} 
%   \begin{block}{Multiple contact notation}
%     For each contact $\alpha \in \{1,\ldots n_c\}$, we have
%     \begin{itemize}
%     \item the local velocity : $u^\alpha \in \nbR^3$, and
%       $$  u = [[u^\alpha]^\top, \alpha = 1\ldots n_c]^\top$$
%     \item the local  reaction vector $r^\alpha\in \nbR^3$    
%       $$  r = [[r^\alpha]^\top, \alpha = 1\ldots n_c]^\top$$
%     \item the local  Coulomb cone    $$K^{\alpha}  = \{r^\alpha, \|r^\alpha_\t \| \leq \mu^\alpha |r^\alpha_\n| \} \subset \nbR^3$$
%        and the set $K$ is the cartesian product of Coulomb's friction cone at each contact, that 
%       \begin{equation}
%         \label{eq:CC}
%         K = \prod_{\alpha=1\ldots n_c} K^{\alpha} 
%       \end{equation}
%       and $K^\star$ is dual.
%     \end{itemize} \end{block}
  
% }





\frame{
  \frametitle{Discrete frictional contact problems} 
  \begin{problem}[General discrete frictional contact problem]\label{prob:I}
  Given
  \begin{itemize}
    \item a symmetric positive definite matrix ${M} \in \nbR^{n \times n}$,
    \item a vector $ {f} \in \nbR^n$,
    \item a matrix  ${H} \in \nbR^{n \times m}$,
    \item a vector $w \in \nbR^{m}$,
    \item a vector of coefficients of friction $\mu \in \nbR^{n_c}$,
  \end{itemize}
find three vectors $ {v} \in \nbR^n$, $u\in\nbR^m$ and $r\in \nbR^m$, denoted by $\mathrm{FC/I}(M,H,f,w,\mu)$  such that
\begin{equation}\label{eq:soccp1}
  \begin{cases}
    M v = {H} {r} + {f} \\[2mm]
    u = H^\top v + w \\[2mm]
    \tilde u = u + g(u) \\[2mm]
    K^\star \ni {\tilde u} \perp r \in K
  \end{cases}
\end{equation}
with $g(u) = [[\mu^\alpha  \|u^\alpha_\t\| {\sf N}^\alpha]^\top, \alpha = 1\ldots n_c]^\top$. 
\qed
\end{problem}

}

\frame{
  \frametitle{Discrete frictional contact problems}
  \only<1>{ \begin{block}{Wide range of applications}
      The problem is:
      \begin{itemize}
      \item is generic enough to include a large number of cases in practice,
      \item is really representative in the linear, or the linearized,  case (Newton procedure),
      \item can be generalised to non-linear cases.
      \end{itemize}
  \end{block}
  See for instance~\parencite{Acary.Cadoux2013}\\
  
 
  \begin{block}{Origin of the linear relation $u = H^\top v + w$}

     \begin{itemize}
     \item $H$ is the contact configuration matrix (similar to the Jacobians of the constraints)
     \item $w$ can contain
       \begin{itemize}
       \item impact laws terms or prescribed velocity in velocity level formulations
       \item displacements, or increments of displacements, in position level formulations
       \end{itemize}
     \end{itemize}
   \end{block}
 }
}
\frame{
  \frametitle{Discrete frictional contact problems}
   \only<1>{
     \begin{block}{Origin of the linear relation $  M v = {H} {r} + {f}$}
       \begin{itemize}
    
       \item Time--discretization of the discrete dynamical mechanical
         system. Event--capturing or event--detecting  time--stepping schemes
       \item Space discretization of the quasi--static problem of solids (FEM)\\
         (M is the tangent stiffness matrix !).
       \item Time--discretization and space discretization of the
         dynamic problem of solids. (FEM, MPM, PFEM, \ldots)

       \item Flexible or rigid  multi-body Systems,
       \item Spectral methods, harmonic balance method, \ldots
       \end{itemize}
     \end{block}
    }
}
\frame{
  \frametitle{Discrete frictional contact problems} 
  \begin{problem}[Reduced discrete frictional contact problem]\label{prob:II}
    Given
    \begin{itemize}
    \item a symmetric positive semi--definite  matrix ${W} \in \nbR^{m \times m}$,
    \item a vector $ {q} \in \nbR^m$,
    \item a vector $\mu \in \nbR^{n_c}$ of coefficients of friction, 
    \end{itemize}
    find two vectors $u\in\nbR^m$ and $r\in \nbR^m$, denoted by $\mathrm{FC/II}(W,q,\mu)$  such that
    \begin{equation}\label{eq:soccp2}
      \begin{cases}
        u =Wr +q \\[2mm]
        \tilde u =u + g(u) \\[2mm]
        K^\star \ni {\tilde u} \perp r \in K
      \end{cases}
    \end{equation}
    with $g(u) = [[\mu^\alpha \|u^\alpha_\t\| {\sf N}^\alpha]^\top,  \alpha = 1\ldots n_c]^\top$.
    \qed
  \end{problem}
  
  \begin{block}{Relation with the general problem}
     $ W = H^\top M^{-1} H$ and $q = H^\top  M^{-1} f + w $.
  \end{block}
  
}
\subsection{From the optimization point of view}
\frame{
  \frametitle{From the optimization point of view} 
%  \begin{block}{Complementarity problem / Variational inequality}
    Discrete frictional contact are complementarity problems / variational inequalities. \\[3mm]
    \quad\tr{Finite dimensional Second-Order Cone Complementarity Problems (SOCCP)}
    \begin{equation}
      \label{eq:soccp3}
      K^\star \ni Wr +q + g(Wr +q) \perp r \in K
    \end{equation}
    of more generally,\\[1mm]
    \quad\tr{Variational Inequality (VI)} (normal cone inclusion)
    \begin{equation}
      \label{eq:inclusion-1}
      -(W r +q + g(Wr+q)) \stackrel{\Delta}= - F(r)  \in N_K(r).
    \end{equation}
%  \end{block}
  \begin{block}{Properties}
    \begin{itemize}
    \item nonsmooth since  $g()$ is nonsmooth
    \item nonmonotone since the mapping $F$ is not monotone for large $\mu$
    \item many possible reformulations such as nonsmooth equations $G(r) = 0$
    \end{itemize}
  \end{block}
}  
\frame{
  \frametitle{From the optimization point of view}

  \begin{block}{Important Remarks}
    \begin{itemize}
    \item The variational inequality is NOT  the optimality condition of a (convex) optimization problem.
    \item The problem is hard to solve efficiently and robustly at tight accuracy.
    \item Even harder if $H$ is not full rank (constraints redundancy) 
    \item Generic numerical methods for VI/CP exist and can be applied
    \item Numerous of existing methods for FC3D problems are adaptations of mathematical programming methods.
    \end{itemize}
  \end{block}
}
\frame{
  \frametitle{From the optimization point of view}
  \begin{block}{Semismooth Newton methods for nonsmooth equations $G(r) = 0$.}
    Not just adaptations, but sometimes pioneering methods.
    \begin{itemize}
    \item The natural map $F^\nat$ associated with the VI (\ref{eq:inclusion-1}) $$ F^\nat(r) = r - P_{K}(r-F(r))$$
    \item Pioneering work of~\cite{Alart.Curnier1991}
      \begin{equation*}
        \label{eq:CKPS-1}
        \begin{cases}
          r_\n - P_{\nbR^{n_c}_+}(r_\n - \rho_\n  u_\n) = 0, \\
          r_\t - P_{D(\mu, r_{\n,+}+\rho u_\n)}(r_\t - \rho_\t u_\t   )=0,
        \end{cases}
      \end{equation*}
     
    \item other SOCCP functions (Fisher-Bursmeister function)
      
    \end{itemize}
  \end{block}
  
 
 }
 \frame{
  \frametitle{From the optimization point of view} 
  \begin{block}{An optimization problem}

    \begin{equation}
      \label{eq:2}
      \begin{array}{lcl}
        \min_{v,u,r} & & \tilde u ^\top r = u^\top r + \mu r_\n \|u_{\t}\|  \stackrel{\Delta}= b(u,r)\\[1mm]
        \text{s.t.} & & Mv = H r + f \\[1mm]
                    & & \tilde u = H^\top v + w + g(u) \in K^\star \\[1mm]
                    & & r \in K 
        
      \end{array}
    \end{equation}
    $b(u,r)$ is the de Saxcé bi-potential.
    \begin{itemize}
    \item A solution of the discrete frictional contact problem is a solution of the optimization problem \eqref{eq:2} with $b(u,r)=0$
    \item A solution of the optimization problem \eqref{eq:2} is a solution of the discrete frictional contact problem if $b(u,r)=0$
    \item With constraints qualification, the problem has a solution.
    \item The problem is not convex and non smooth, may have a lot of local minima.
 
    \end{itemize}
    \ding{220} In practice,  finding a minimum is difficult, and a global minimum is not ensured.
   
  \end{block}

}
  

 


%%% Local Variables:
%%% mode: latex
%%% TeX-master: "s"
%%% End:

\frame{\thispagestyle{empty}\tableofcontents\addtocounter{framenumber}{-1}}


\section{An existence result}
\label{Sec:existence}

\frame{
  \frametitle{An existence result. (F. Cadoux PhD)}
  Let us introduce a slack variable
  \begin{equation} \label{eq:chgVarDS}
    s^\alpha := \| u^\alpha_\t \|
  \end{equation}
  together with a new modified velocity
  $$ \widetilde u := u + \alpha s $$ $g(u) = \alpha s$
  The problem $\mathrm{FC/I}(M,H,f,w,\mu)$ can be reformulated as
  \begin{equation}\label{eq:soccp1-convex}
  \begin{cases}
    M v = {H} {r} + {f} \\[2mm]
    \widetilde  u = H^\top v + w  + \alpha s  \\[2mm]
    K^\star \ni {\hat u} \perp r \in K
  \end{cases}
\end{equation}
}
\frame{
  \frametitle{An existence result.}
  The problem~(\ref{eq:soccp1-convex}) appears to be the KKT condition of
  \begin{block}{primal problem}
    \begin{equation} \tag{$D_s$}
      \left\{
        \begin{array}{l}
          \min \quad J(v) := \frac{1}{2} v^\top M v + f^\top v \\
          H^\top v + w + \alpha s \in K^\star
        \end{array}
      \right.
    \end{equation}
  \end{block}
  \begin{block}{dual problem}
    \begin{equation} \tag{$P_s$}
      \left\{
        \begin{array}{l}
          \min \quad J_s(r) := \frac{1}{2} r^\top W r - q_s^\top r \\
          r \in K
        \end{array}
      \right.
    \end{equation}
    with $q_s = q + \alpha s $
  \end{block}
  \begin{block}{Interest}
    Two convex program \ding{220} existence of solutions under feasibility conditions.
  \end{block}
}
\frame{
  \frametitle{An existence result.}
  \begin{block}{Fixed point problem}
    Introducing
  $$ u(s) := \argmin_u (P_s) = \argmin_u (D_s) $$
  practically \alert{computable} by optimization software, and
  $$ F^i(s) := \| u^i_T(s) \|, $$
  the incremental problem becomes a fixed point problem
  $$ F(s) = s $$
\end{block}

}
\frame{
  \frametitle{An existence result.}

  \begin{block}{Assumption}
    $$ \exists v \in \R^m \ : \ H v + w \in K^* $$
  \end{block}
  Using the assumption,
  \begin{itemize}
  \item the application $F: \R_+^n \rightarrow \R_+^n$ is \alert{well-defined}, \alert{continuous} and \alert{bounded}
  \item apply Brouwer's theorem
  \end{itemize}
  \begin{theorem}
    A fixed point exists
  \end{theorem}
  This result is a variant of a previous result obtained by~\cite{Klarbring-Pang-98}.
}
\frame{
  \frametitle{An existence result.}
  \begin{block}{Numerical validation of the assumption}
    The assumption by solving a linear program over of a SOC.
  \end{block}
  \begin{block}{Numerical interest}
    The fixed point equation $F(s) = s$ can be tackled by
    \begin{itemize}
    \item \alert{fixed-point} iterations
      $$ s \leftarrow F(s) $$
    \item \alert{Newton} iterations
      $$s \leftarrow \mathrm{Jac}[F](s) \backslash F(s) $$
    \item Variants possible (truncated resolution of inner problem\dots)
    \end{itemize}
  \end{block}
}


%%% Local Variables: 
%%% mode: latex
%%% TeX-master: "s"
%%% End: 

\frame{\thispagestyle{empty}\tableofcontents\addtocounter{framenumber}{-1}}



\section{Numerical solution procedure.}
\label{Sec:Numerics}
\subsection{VI based methods}

\frame{
  \frametitle{VI based methods}
  \begin{itemize}
  \item Basic fixed point iterations with projection
    $$\sf z_{k+1} \leftarrow P_{X}(z_k - \rho_k\,F(z_k))$$
  \item Extragradient method
    $$\sf z_{k+1} \leftarrow P_X(z_{k}-\rho_k\,F(P_X(z_k-\rho_k F(z_k))))$$
  \item Hyperplane projection method
  \end{itemize}
  plus self-adaptive procedure for $\rho_k = \rho 2^{m_k}$:
  \begin{equation}
    \label{eq:Korpelevitch}
    \rho_k \| F(z_k)-F(\bar z_k) \| \leq \|z_k-\bar z_k\|
  \end{equation}
  
}


\subsection{Nonsmooth Equations based methods}
\frame{
  \frametitle{Nonsmooth Equations based methods}
  Nonsmooth Newton on   $G(z)=0$
   $$z_{k+1}  =  z_k -  \Phi^{-1}(z_k) (G(z_k)), \Phi(z_k) \in \partial G(z_k)$$
   \begin{itemize}
   \item Alart--Curnier Formulation~\cite{Alart.Curnier1991}
     \begin{equation}
       \label{eq:CKPS-1}
       \begin{cases}
         r_\n - P_{\RR^{n_c}_+}(r_\n - \rho_\n  u_\n) = 0, \\
         r_\t - P_{D(\mu, r_{\n,+})}(r_\t - \rho_\t u_\t   )=0,
       \end{cases}
     \end{equation}
   \item Direct normal map reformulation
     $$     r - P_{K}\left(r  - \rho (u   + g(u))\right) = 0$$
   \item  Extension of  Fischer-Burmeister function to SOCCP
     $$\phi_{\fb}(x,y) = x+y - (x^2 + y^2 )^{1/2}$$
     with Jordan product and square root
   \end{itemize}
}

\subsection{Matrix Splitting and projection based algorithms}
\frame{
  \frametitle{Matrix Splitting and projection based algorithms}

}





\subsection{Optimization based approach}
\frame{
  \frametitle{ Optimization based methods}
  \begin{itemize}
  \item Successive approximation with Tresca friction (Haslinger et al.)
    \begin{equation}
      \label{eq:Haslinger-2}
      \begin{cases}
        \theta = h(r_\n) \\[2mm]
        \min\, \Frac 1 2 r^\top W r + r^\top q \\
        \begin{array}{ll}
          \text{s.t. }& r \in C(\mu,\theta)
        \end{array}
      \end{cases}
    \end{equation}
    where $C(\mu,\theta)$ is the cylinder of radius $\mu\theta$.
  \item Fixed point on the norm of the tangential velocity~[A., Cadoux, Lemar\'echal, Malick(2011)]~\nocite{ZAMM:ZAMM201000073}.
    \begin{equation}\label{eq:ACLM-3}
      \begin{cases}
        s =  \| u_\t \| \\[2mm]
        \min\,\Frac 1 2 r^\top W r + r^\top (q + \alpha s)  \\
        \begin{array}{ll}
          \text{s.t. } & r \in K
        \end{array}
      \end{cases}
    \end{equation}
    Fixed point or Newton Method on $F(s)=s$
  \item  Alternating optimization problems (Panagiotopoulos et al.)
  \end{itemize}
}
\subsection{Siconos/Numerics}
\frame{
\frametitle{Siconos/Numerics}
\begin{block}
  {\sc Siconos} Open source software for modelling and simulation of
  nonsmooth systems
\end{block}
\begin{block} {\sc Siconos/Numerics} Collection of C routines to
  solve FC3D problem
  \begin{itemize}
  \item NonSmoothGaussSeidel : VI based projection/splitting
    algorithm
  \item TrescaFixedPoint : fixed point algorithm on Tresca fixed
    point
  \item LocalAlartCurnier : semi--smooth newton method of
    Alart--Curnier formulation
  \item ProximalFixedPoint : proximal point algorithm
  \item VIFixedPointProjection : VI based fixed-point projection
   \item VIExtragradient : VI based extra-gradient method
   \item \ldots
   \end{itemize}
 \end{block}
 \begin{block}{\url{http://siconos.gforge.inria.fr}}
   use and contribute ...
 \end{block}
  
}







%%% Local Variables: 
%%% mode: latex
%%% TeX-master: "s"
%%% End: 

\frame{\thispagestyle{empty}\tableofcontents\addtocounter{framenumber}{-1}}

%\section{Siconos/Numerics}
\frame{
  \frametitle{Siconos/Numerics}
  \begin{block}
    {\sc Siconos} Open source software for modelling and simulation of
    nonsmooth systems
  \end{block}
  \begin{block} {\sc Siconos/Numerics}
    Collection of C routines to solve FC3D problems in dense, sparse or  block sparse versions:
    \begin{itemize}
    \item VI solvers: Fixed point, Extra-Gradient, Uzawa
    \item VI based projection/splitting algorithm: NSGS, PSOR
    \item Semismooth Newton methods:\\
      Alart-Curnier, Jean-Moreau, Natural map, Ficher-Bursmeister
    \item Proximal point algorithm
    \item Optimization based solvers. Panagiotopoulos, Tresca, SOCQP
    \item \tr{Interior point methods}, \ldots
    \end{itemize}
  \end{block}
  \begin{block}{Collection of routines for optimization and complementarity problems}
    \begin{itemize}
    \item LCP solvers (iterative and pivoting (Lemke))
    \item Standard QP solvers (Projected Gradient (Calamai \& Mor\'e), Projected CG (Mor\'e \& Toraldo), active set technique)
    \item linear and nonlinear programming solvers.
    \end{itemize}
  \end{block}
}
\begin{frame}[fragile]
  \frametitle{Siconos/Numerics}
  \begin{block}
    {Implementation details}
     \begin{itemize}
     \item Matrix format.
       \begin{itemize}
       \item dense (column-major)
       \item sparse matrices (triplet, CSR, CSC)
       \end{itemize}
     \item Linear algebra libraries and solvers.
       \begin{itemize}
       \item BLAS/LAPACK, MKL
       \item MUMPS, SUPERLU, UMFPACK,
       \item PETSc
       \end{itemize}
     \item Python interface (swig (pybind11 coming soon))
     \item Generic structure for problem, driver and options
     \end{itemize}
   \end{block}
        {\small
        \begin{minted}{c}
          int fc3d_driver(FrictionContactProblem* problem,
                          double* reaction,
                          double* velocity,
                          SolverOptions* numerics_solver_options);
       \end{minted}
     }
\end{frame}



% \frame{
%   \frametitle{Siconos/Numerics}

  
% }
\begin{frame}[fragile]
  \frametitle{C structure to encode the problem} 
  {
    \begin{block}{Reduced discrete frictional contact problem}
      {\small
        \begin{minted}{cpp}
          struct FrictionContactProblem {
            /** dimension of the contact space (3D or 2D ) */
            int dimension;
            /** the number of contacts \f$ n_c \f$ */
            int numberOfContacts;
            /** \f$ {M} \in {{\mathrm{I\!R}}}^{n \times n} \f$,
            a matrix with \f$ n = d  n_c \f$ stored in NumericsMatrix structure */
            NumericsMatrix *M;
            /** \f$ {q} \in {{\mathrm{I\!R}}}^{n} \f$ */
            double *q;
            /** \f$ {\mu} \in {{\mathrm{I\!R}}}^{n_c} \f$, vector of friction coefficients
            (\f$ n_c = \f$ numberOfContacts) */
            double *mu;
          };
        \end{minted}
      }
    \end{block}
  }
\end{frame}
\begin{frame}[fragile]
  \frametitle{C structure to encode the problem}
  % \only<1>
  {
    \begin{block}{Global discrete frictional contact problem}
      {\small
        \begin{minted}{cpp}
          struct GlobalFrictionContactProblem {
            /** dimension \f$ d=2 \f$ or \f$ d=3 \f$ of the contact space (3D or 2D ) */
            int dimension;
            /** the number of contacts \f$ n_c \f$ */
            int numberOfContacts;
            /** \f$ M \in {\mathrm{I\!R}}^{n \times n} \f$,
            a matrix with \f$ n\f$ stored in NumericsMatrix structure */
            NumericsMatrix *M;
            /**  \f$ {H} \in {{\mathrm{I\!R}}}^{n \times m} \f$,
            a matrix with \f$ m = d  n_c\f$ stored in NumericsMatrix structure */
            NumericsMatrix *H;
            /** \f$ {q} \in {{\mathrm{I\!R}}}^{n} \f$ */
            double *q;
            /** \f$ {b} \in {{\mathrm{I\!R}}}^{m} \f$ */
            double *b;
            /** \f$ {\mu} \in {{\mathrm{I\!R}}}^{n_c} \f$, vector of friction
            coefficients
            (\f$ n_c = \f$ numberOfContacts) */
            double *mu;
          };
        \end{minted}
      }
    \end{block}
  }
\end{frame}

\begin{frame}[fragile]
  \frametitle{A very basic example in C}
    {\small
      \begin{minted}{cpp}
// Problem Definition
int NC = 3;//Number of contacts
double M[81] = {1, 0, 0, 0, 0, 0, 0, 0, 0, 0, 1, 0, 0, 0, 0, 0, 0, 0, 0, 0, 1, 0, 0, 0, 0, 0, 0, 0, 0, 0, 1, 0, 0, 0, 0, 0, 0, 0, 0, 0, 1, 0, 0, 0, 0, 0, 0, 0, 0, 0, 1, 0, 0, 0, 0, 0, 0, 0, 0, 0, 1, 0, 0, 0, 0, 0, 0, 0, 0, 0, 1, 0, 0, 0, 0, 0, 0, 0, 0, 0, 1};
double q[9] = { -1, 1, 3, -1, 1, 3, -1, 1, 3};
double mu[3] = {0.1, 0.1, 0.1};

FrictionContactProblem NumericsProblem;
NumericsProblem.numberOfContacts = NC;
NumericsProblem.dimension = 3;
NumericsProblem.mu = mu;
NumericsProblem.q = q;

NumericsMatrix *MM = (NumericsMatrix*)malloc(sizeof(NumericsMatrix));
MM->storageType = NM_DENSE;
MM->matrix0 = M;
MM->size0 = 3 * NC;
MM->size1 = 3 * NC;
NumericsProblem.M = MM;
\end{minted}
}
\end{frame}
  
\begin{frame}[fragile]
  \frametitle{A basic example in C}
    {\small
      \begin{minted}{cpp}
// Variable declaration
double *reaction = (double*)calloc(3 * NC, sizeof(double));
double *velocity = (double*)calloc(3 * NC, sizeof(double));

// Numerics and Solver Options
SolverOptions *numerics_solver_options = solver_options_create(SICONOS_FRICTION_3D_NSGS);
numerics_solver_options->iparam[SICONOS_IPARAM_MAX_ITER] = 1000;
numerics_solver_options->dparam[SICONOS_DPARAM_TOL] = 100*DBL_EPSILON;
// numerics_set_verbose(2);

// Driver call
fc3d_driver(&NumericsProblem,
            reaction, velocity,
            numerics_solver_options);
        
      \end{minted}
    }
  \end{frame}

\begin{frame}[fragile]
  \frametitle{A basic example in Python}
    {\small
      \begin{minted}{python}
import numpy as np
import siconos.numerics as sn


NC = 1
M = np.eye(3 * NC)
q = np.array([-1.0, 1.0, 3.0])
mu = np.array([0.1])
FCP = sn.FrictionContactProblem(3, M, q, mu)


reactions = np.array([0.0, 0.0, 0.0])
velocities = np.array([0.0, 0.0, 0.0])
sn.numerics_set_verbose(1)

      \end{minted}
    }
  \end{frame}
\begin{frame}[fragile]
  \frametitle{A basic example in Python}
    {\small
      \begin{minted}{python}
def solve(problem, solver, options):
    """Solve problem for a given solver"""
    reactions[...] = 0.0
    velocities[...] = 0.0
    r = solver(problem, reactions, velocities, options)
    assert options.dparam[sn.SICONOS_DPARAM_RESIDU] < options.dparam[sn.SICONOS_DPARAM_TOL]
    assert not r

def test_fc3dnsgs():
    """Non-smooth Gauss Seidel, default"""
    SO = sn.SolverOptions(sn.SICONOS_FRICTION_3D_NSGS)
    solve(FCP, sn.fc3d_nsgs, SO)

def test_fc3dlocalac():
    """Non-smooth Gauss Seidel, Alart-Curnier as local solver."""
    SO = sn.SolverOptions(sn.SICONOS_FRICTION_3D_NSN_AC)
    solve(FCP, sn.fc3d_nonsmooth_Newton_AlartCurnier, SO)

def test_fc3dfischer():
    """Non-smooth Newton, Fischer-Burmeister."""
    SO = sn.SolverOptions(sn.SICONOS_FRICTION_3D_NSN_FB)
    solve(FCP, sn.fc3d_nonsmooth_Newton_FischerBurmeister, SO)

if __name__ == "__main__":
    test_fc3dnsgs()
    test_fc3dlocalac()
    test_fc3dfischer()
    

      \end{minted}
    }
  \end{frame}







  
\frame{
  \frametitle{Siconos/Numerics}
  \begin{block}{\url{http://siconos.gforge.inria.fr}}
    use and contribute ...
  \end{block}
}

%%% Local Variables:
%%% mode: latex
%%% TeX-master: "s"
%%% End:


%\section{FCLIB : a collection of discrete 3D Frictional Contact (FC) problems}

\frame{
  \frametitle{FCLIB : a collection of discrete 3D Frictional Contact (FC) problems}
  \begin{itemize}
  \item Few mathematical results: existence, uniqueness, convergence, rate of convergence. 
  \item Our inspiration: MCPLIB or CUTEst in Optimization.
  \item Without convergence proof, test your method on a large set of  benchmarks shared by the community.
  \end{itemize}
  \begin{block}
    {What is FCLIB ?}
    \begin{itemize}
    \item A open source collection of Frictional Contact (FC) problems
      stored in a specific HDF5 format 
    \item A open source light implementation of Input/Output functions
      in C Language to read and write problems (Python and Matlab coming soon)
    \end{itemize}
  \end{block}


  \begin{block}
    {Goals of the project}
    \begin{itemize}
    \item Provide a standard framework for testing available and new algorithms for solving discrete frictional contact problems
    \item Share common formulations of problems in order to exchange data in a reproducible manner.
    \end{itemize}
\end{block}
}

\def\ssep{1.5mm}

\frame{
  
\def\figillus{0.15\textheight}
\begin{figure}
  \centering
  \subfloat[Cubes\_H8]{\includegraphics[height=\figillus]{figure/Cubes_H8_5.png}$\quad$}
  \subfloat[LowWall\_FEM]{\includegraphics[height=\figillus]{figure/LowWall_FEM.png}}
  \subfloat[Aqueduct\_PR]{\includegraphics[height=\figillus]{figure/Aqueduc_PR.png}$\quad$}
  \subfloat[Bridge\_PR]{\includegraphics[height=\figillus]{figure/Bridge_PR_1.png}}\\
  \subfloat[100\_PR\_Periobox]{\includegraphics[height=\figillus]{figure/100_PR_PerioBox.png}$\quad$}
  \subfloat[945\_SP\_Box\_PL]{\includegraphics[height=\figillus]{figure/945_SP_Box_PL.png}}
  \subfloat[Capsules]{\includegraphics[height=\figillus]{figure/Capsules.png}$\quad$}
  \subfloat[Chain]{\includegraphics[height=\figillus]{figure/Chains.png}$\quad$}\\
  \subfloat[KaplasTower]{\includegraphics[height=\figillus]{figure/KaplasTower.png}$\quad$}
  \subfloat[BoxesStack]{$\quad$$\quad$\includegraphics[height=\figillus]{figure/BoxesStack.png}$\quad$$\quad$}
  \subfloat[Chute\_1000, Chute\_4000, Chute\_local\_problems]{$\quad$$\quad$$\quad$$\quad$\includegraphics[height=\figillus]{figure/Chute_1000_light.jpg}$\quad$$\quad$$\quad$$\quad$}
  \caption{Illustrations of the FClib test problems from Siconos and LMGC90}
  \label{fig:fclib}
\end{figure}
}

% \begin{frame}
%   \scriptsize
%   \begin{table}
%   \newcolumntype{H}{>{\setbox0=\hbox\bgroup}c<{\egroup}@{}} \begin{tabular}{|l|l|l|l|l|l|l|l|H@{\hspace*{-\tabcolsep}}|H@{\hspace*{-\tabcolsep}}|H@{\hspace*{-\tabcolsep}}|}
%   \hline
%   Test set
%   & code
%   & \parbox[t]{2mm}{\rotatebox[origin=c]{-90}{  friction coefficient $\mu$  }}
%   & \parbox[t]{2mm}{\rotatebox[origin=c]{-90}{ \# of problems }}
%   & \parbox[t]{2mm}{\rotatebox[origin=c]{-90}{ \# of d.o.f. }}
%   & \parbox[t]{2mm}{\rotatebox[origin=c]{-90}{ \# of contacts }}
%   & \parbox[t]{2mm}{\rotatebox[origin=c]{-90}{contact density $c$}}
%   %& \parbox[t]{2mm}{\rotatebox[origin=c]{-90}{rank(W)}}
%   & \parbox[t]{2mm}{\rotatebox[origin=c]{-90}{rank ratio(W)}}
%   & \parbox[t]{2mm}{\rotatebox[origin=c]{-90}{cond(W)}}
%   & \parbox[t]{2mm}{\rotatebox[origin=c]{-90}{cond(W) LSMR}}
%   & \parbox[t]{2mm}{\rotatebox[origin=c]{-90}{$\mu(\|W-W^T\|)$}}
%     \parbox[t]{2mm}{\rotatebox[origin=c]{-90}{symmetry of $W$}}
%   \\
%   \hline
%   \hline
%   Cubes\_H8\_2
%   & LMGC90
%   & 0.3
%   & 15
%   & 162
%   & $[3:5]$
%   & $[0.02:0.09]$
% %  & $[6:15]$
%   & 1
%   & $[2.2.10^{1}:1.3.10^{3}]$
%   & $[8.1.10^{5}:1.5.10^{6}]$
%   & $3.2.10^{-4}$\\
%    \hline
%   Cubes\_H8\_5
%   & LMGC90
%   & 0.3
%   & 50
%   & 1296
%   & $[17:36]$
%   & $[0.02:0.09]$
% %  & $[48:108]$
%   & 1
%   & $[3.3.10^{4}: 7.2.10^{4} ]$
%   & $[1.3.10^{6}: 3.1.10^{6} ]$
%   & $4.2.10^{-4}$\\
%    \hline
%   Cubes\_H8\_20
%   & LMGC90
%   & 0.3
%   & 50
%   & 55566
%   & $[361:388]$
%   & $[0.019:0.021]$
% %  & $[1083:1164]$
%   & 1
%   & $[2.4.10^{5}: 2.5.10^{5} ]$
%   & $[1.3.10^{6}: 5.2.10^{6} ]$
%   &  $5.2.10^{-5}$\\
%   \hline
%   LowWall\_FEM
%    & LMGC90
%   & 0.83
%   & 50
%   & \{7212\}
%   & $[624:688]$
%   & $[0.28:0.29]$
% %  & $[1873:2064]$
%   & 1
%   & --
%   & $[9.3.10^{2}:5.0.10^{5}]$
%   &  $5.2.10^{-2}$\\
%   \hline
%   Aqueduct\_PR
%   & LMGC90
%   & 0.8
%   & 10
%   & \{1932\}
%   & $[4337:4811]$
%   & $[6.81:7.47]$
% %  & $[1934]$$
%   & $[6.80:7.46]$
%   & $[4.7.10^{7}:3.4.10^{8}]$
%   & $[6.7.10^{1}:1.5.10^{2}]$
%   &  $1.1.10^{-15}$\\
%   \hline
%   Bridge\_PR
%   & LMGC90
%   & 0.9
%   & 50
%   & \{138\}
%   & $[70:108]$
%   & $[1.5:2.3]$
% %  & $[87:132]$
%   & $[2.27:2.45]$
%   & $[8.3.10^{4}:1.1.10^{5}]$
%   & $[1.9.10^{3}:2.6.10^{4}]$
%   & $5.8.10^{-18}$\\
%   \hline
%   100\_PR\_Periobox
%   & LMGC90
%   & 0.8
%   & 106
%   & \{606\}
%   & $[14:578]$
%   & $[0.2:3]$
% %  & $[23:600]$
%   & $[1.76:3.215]$
%   & $[4.3.10^{2}:1.0.10^{6}]$
%   & $[6.3.10^{5}:3.5.10^{6}]$
%   & $8.8.10^{-20}$\\
%   \hline
%   945\_SP\_Box\_PL
%   & LMGC90
%   & 0.8
%   & 60
%   & \{5700\}
%   & $[2322:5037]$
%   & $[1.22:2.65]$
% %  & $[5333:7617]$
%   & $[1.0:2.66]$
%   & $[2.2.10^{4}:4.4.10^{5}]$
%   & $[2.9.10^{1}:9.2.10^{2}]$
%   & $1.3.10^{-10}$\\
%   \hline
%   Capsules
%   & Siconos
%   & 0.7
%   & 249
%   & [96:600]
%   & $[17:304]$
%   & $[0.53:1.52]$
% %  & $[47:587]$
%   & $[1.08:1.55]$
%   & --
%   & $[4.8:1.6.10^{2}]$
%   & $3.3.10^{-02}$\\
%   \hline
%   Chain
%   & Siconos
%   & 0.3
%   & 242
%   & \{60\}
%   & $[8:28]$
%   & $[0.5:1.3]$
% %  & 53
%   & $[1.05:1.6]$
%   & $[7.4.10^{4}:4.0.10^{9}]$
%   & $[1.5.10^{1}:4.7.10^{5}]$
%   & $3.7.10^{-02}$\\
%   \hline
%   KaplasTower
%   & Siconos
%   & 0.7
%   & 201
%   & $[72:792]$
%   & $[48:933]$
%   & $[3.0:3.6]$
% %  & $[72:792]$
%   & $[2.0:3.53]$
%   & $[67:2174]$
%   & $[8:67]$
%   & $5.4.10^{-08}$\\
%   \hline
%   BoxesStack
%   & Siconos
%   & 0.7
%   & 255
%   & $[6:300]$
%   & $[1:200]$
%   & $[1.86:2.00]$
% %  & $[54:300]$
%   & $[1.875:2.0]$
%   & $[3.8.10^{4}:2.5.10^{7}]$
%   & $[9.0:5.4.10^{3}]$
%   & $2.23.10^{-14}$\\
%   \hline
%   Chute\_1000
%   & Siconos
%   & 1.0
%   & 156
%   & $[276:5508]$
%   & $[74:5056]$
%   & $[0.69:2.95]$
% %  & $[222:8133]$
%   & $[1.0:2.95]$
%   & $[2.1.10^{1}:1.9.10^{3}]$
%   & $6.6.10^{-02}$ \\
%   \hline
%   Chute\_4000
%   & Siconos
%   & 1.0
%   & 40
%   & $[17280:20034]$
%   & $[15965:19795]$
%   & $[2.51:3.06]$
%   & --
%   % & $[222:8133]$
%   & --
%   & $[5.5.10^{1}:9.0.10^{3}]$
%   & $8.9.10^{-14}$\\
%   \hline
%   Chute\_local\_problems
%   & Siconos
%   & 1.0
%   & 834
%   & 3
%   & 1
%   & 1
% %  & 3
%   & 1
%   & $[1.04:4.66]$
%   & $[2.6:2.6.10^{1}]$
%   & $1.76.10^{-09}$\\
%   \hline
% \end{tabular}
% \caption{Description of the test sets of FCLib library (v1.0)}
% \label{Tab:fclib}
% \end{table}
% \end{frame}

%%% Local Variables:
%%% mode: latex
%%% TeX-master: "s"
%%% End:



\section{Preliminary  Comparisons}
\label{Sec:Comparison}
\subsection{Performance profiles}
\frame{
  \frametitle{Performance profiles~\cite{Dolan.More2002}}

  \begin{itemize}
  \item Given a set of problems $\mathcal P$
  \item Given a set of solvers $\mathcal S$  
  \item A performance measure for each problem  with a solver $t_{p,s}$ (cpu time, flops, ...)
  \item Compute the performance ratio
    \begin{equation}
      \label{eq:perf-ratio}
      \tau_{p,s} =    \Frac{t_{p,s}}{\min_{s\in\mathcal S} t_{p,s}} \geq 1
    \end{equation}
  \item Compute the performance profile $\rho_s(\tau) : [1,+\infty]\rightarrow [0,1]$ for each solver $s\in \mathcal S$
    
    \begin{equation}
      \rho_s(\tau) = \Frac{1}{|\mathcal P|}\big|\{p\in \mathcal P\mid \tau_{p,s} \leq \tau    \}\big|\label{eq:perf}
  \end{equation}
  The value of $\rho_s(1)$ is the probability that the solver $s$ will win over the rest of the solvers.
  \end{itemize}
  
  
}
\def\ssep{1.5mm}

\subsection{Chain}
\frame{
  \frametitle{First comparisons. Chain}
  \begin{block}
    {Hanging chain with initial velocity at the tip}
    code: Siconos
    $$ $$
    \begin{minipage}{0.39\linewidth}
      \includegraphics[width=1.0\textwidth]{Chains}
    \end{minipage}
    \begin{minipage}{0.49\linewidth}
      \begin{tabular}{|p{0.7\textwidth}|c|}
        coefficient of friction & $0.3$ \\[\ssep]
        number of problems & 1514 \\[\ssep]
        number of degrees of freedom & [48 : 60] \\[\ssep]
        number of contacts & [8 :28] \\[\ssep]
        required accuracy   & $10^{-8}$    
      \end{tabular}
    \end{minipage}
  \end{block}

}
\frame{
  \frametitle{First comparisons. Chain}
  \includegraphics[width=1.10\textwidth]{distrib-Chain.pdf}
}
\frame{
  \frametitle{First comparisons. Chain}
    \centerline{\includegraphics[width=0.7\textwidth]{COMP/large/flpops/profile-Chain.pdf}}
    \centerline{\includegraphics[width=0.7\textwidth]{COMP/large/flpops/profile-Chain_legend.pdf}}
}

\subsection{Capsules}

\frame{
  \frametitle{First comparisons. Capsules}
 \begin{block}
    {100 capsules dropped into a box.}
    code: Siconos
    $$ $$
  \begin{minipage}{0.49\linewidth}
    \includegraphics[width=1.0\textwidth]{Capsules}
  \end{minipage}  
  \begin{minipage}{0.49\linewidth}
    \begin{tabular}{|p{0.7\textwidth}|c|}
      coefficient of friction & $0.7$ \\[\ssep]
      number of problems & 1705 \\[\ssep]
      number of degrees of freedom & [6 : 600] \\[\ssep]
      number of contacts &  [0:300]\\[\ssep]
      required accuracy   & $10^{-8}$    
    \end{tabular}
  \end{minipage}
\end{block}
}
\frame{
  \frametitle{First comparisons. Capsules}
  \includegraphics[width=1.10\textwidth]{distrib-Capsules.pdf}
}
% \frame{
%   \frametitle{First comparisons. Capsules}
%   \centerline{\includegraphics[width=1.10\textwidth]{profile-Capsules.pdf}}
% }
\frame{
  \frametitle{First comparisons. Capsules}
    \centerline{\includegraphics[width=0.7\textwidth]{COMP/large/flpops/profile-Capsules.pdf}}
    \centerline{\includegraphics[width=0.7\textwidth]{COMP/large/flpops/profile-Capsules_legend.pdf}}
}




\subsection{Performance profiles. BoxesStack}

\frame{
  \frametitle{First comparisons. BoxesStack}
  \begin{block}
    {50 boxes stacked under gravity.}
    code: Siconos
    $$ $$
  \begin{minipage}{0.14\linewidth}
    \includegraphics[width=1.0\textwidth]{BoxesStack}
  \end{minipage}
  \begin{minipage}{0.25\linewidth}
    \includegraphics[width=1.0\textwidth]{BoxesStack2}
  \end{minipage}
  \begin{minipage}{0.49\linewidth}
    \begin{tabular}{|p{0.7\textwidth}|c|}
      coefficient of friction &  0.7\\[\ssep]
      number of problems &  1159 \\[\ssep]
      number of degrees of freedom & [6 : 300] \\[\ssep]
      number of contacts &  [ 0: 200]\\[\ssep]
      required accuracy   & $10^{-8}$
    \end{tabular}
  \end{minipage}
\end{block}
}
\frame{
  \frametitle{First comparisons. BoxesStack}
  \includegraphics[width=1.10\textwidth]{distrib-BoxesStack1.pdf}
}
% \frame{
%   \frametitle{First comparisons. BoxesStack}
%   \centerline{\includegraphics[width=1.1\textwidth]{profile-BoxesStack1.pdf}}
% }
\frame{
  \frametitle{First comparisons. BoxesStack1}
    \centerline{\includegraphics[width=0.7\textwidth]{COMP/large/flpops/profile-BoxesStack1.pdf}}
    \centerline{\includegraphics[width=0.7\textwidth]{COMP/large/flpops/profile-BoxesStack1_legend.pdf}}
}


\subsection{Performance profiles. Kaplas}

\frame{
  \frametitle{A tower of Kaplas}
  \begin{block}
    {A Tower of Kaplas}
    code: Siconos
    $$ $$
  \begin{minipage}{0.50\linewidth}
    \includegraphics[width=1.0\textwidth]{KaplasTower}
  \end{minipage}
  \begin{minipage}{0.49\linewidth}
    \begin{tabular}{|p{0.7\textwidth}|c|}
      coefficient of friction &  0.3\\[\ssep]
      number of problems &  201 \\[\ssep]
      number of degrees of freedom & [72 : 864] \\[\ssep]
      number of contacts &  [ 0: 950]\\[\ssep]
      required accuracy   & $10^{-8}$
    \end{tabular}
  \end{minipage}
\end{block}
}
\frame{
  \frametitle{A tower of Kaplas}
  \includegraphics[width=1.10\textwidth]{distrib-KaplasTower.pdf}
}
% \frame{
%   \frametitle{A tower of  Kaplas}
%   \centerline{\includegraphics[width=1.1\textwidth]{profile-KaplasTower.pdf}}
% }
\frame{
  \frametitle{First comparisons. Kaplas Tower}
    \centerline{\includegraphics[width=0.7\textwidth]{COMP/large/flpops/profile-KaplasTower.pdf}}
    \centerline{\includegraphics[width=0.7\textwidth]{COMP/large/flpops/profile-KaplasTower_legend.pdf}}
}
% \subsection{Performance profiles. AqueducPR}

% % \frame{An aqueduct}
% %   \begin{block}
% %     {An aqueduct}
% %     code: LMGC
% %     $$ $$
% %   \begin{minipage}{0.50\linewidth}
% %    % \includegraphics[width=1.0\textwidth]{Aqueduc_PR.png}
% %   \end{minipage}
% %   \begin{minipage}{0.49\linewidth}
% %     \begin{tabular}{|p{0.7\textwidth}|c|}
% %       coefficient of friction &  0.5\\[\ssep]
% %       number of problems &  10 \\[\ssep]
% %       number of degrees of freedom & 1932 \\[\ssep]
% %       number of contacts &  [ 4387: 4477]\\[\ssep]
% %       required accuracy   & $10^{-4}$
% %     \end{tabular}
% %   \end{minipage}
% % \end{block}
% % }
% % \frame{
% %   \frametitle{An aqueduct}
% %   \includegraphics[width=1.10\textwidth]{distrib-LMGC_AqueducPR.pdf}
% % }
% % \frame{
% %   \frametitle{A tower of  Kaplas}
% %   \centerline{\includegraphics[width=1.1\textwidth]{profile-KaplasTower.pdf}}
% % }
% \frame{
%   \frametitle{First comparisons. An aqueduct}
%     \centerline{\includegraphics[width=0.7\textwidth]{COMP/large/flpops/profile-LMGC_AqueducPR.pdf}}
%     \centerline{\includegraphics[width=0.7\textwidth]{COMP/large/flpops/profile-LMGC_AqueducPR_legend.pdf}}
% }





\subsection{Performance profiles.  FEM Cube H8}

\frame{
  \frametitle{Two elastic Cubes with FEM discretization H8}
  \begin{block}
    {Two elastic Cubes with FEM discretization H8}
    code : LMGC90
    $$ $$
   \begin{minipage}{0.40\linewidth}
     \includegraphics[width=1.0\textwidth]{Cubes_H8_5}
   \end{minipage}
   \begin{minipage}{0.49\linewidth}
     \begin{tabular}{|p{0.7\textwidth}|c|}
       coefficient of friction &  0.3\\[\ssep]
       number of problems &  58 \\[\ssep]
       number of degrees of freedom & \{162,1083,55566\} \\[\ssep]
       number of contacts &  [ 3:5] [30:36]  [360:368 ]\\[\ssep]
       required accuracy   & $10^{-5}$
     \end{tabular}
  \end{minipage}
\end{block}
}
\frame{
  \frametitle{Two elastic Cubes with FEM discretization H8}
  \includegraphics[width=1.10\textwidth]{distrib-LMGC_Cubes_H8_5.pdf}
 }
% \frame{
%   \frametitle{A tower of  Kaplas}
%   \centerline{\includegraphics[width=1.1\textwidth]{profile-KaplasTower.pdf}}
% }
\frame{
  \frametitle{First comparisons. Cubes H8}
    \centerline{\includegraphics[width=0.7\textwidth]{COMP/large/flpops/profile-LMGC_Cubes_H8.pdf}}
    \centerline{\includegraphics[width=0.7\textwidth]{COMP/large/flpops/profile-LMGC_Cubes_H8_legend.pdf}}
}



%%% Local Variables:
%%% mode: latex
%%% TeX-master: "s"
%%% End:



\frame{\thispagestyle{empty}\tableofcontents\addtocounter{framenumber}{-1}}

\section{Interior Point Methods (IPM)}
\label{Sec:ipm}

% InteriorPoint_QP_beamer.tex
% 4-slide Beamer presentation explaining interior-point techniques for a convex quadratic program.
% Save as InteriorPoint_QP_beamer.tex and compile with pdflatex (or lualatex/xelatex).



% Slide 1: Problem statement + KKT
\begin{frame}{A Primer on IPM for convex QP}
  \begin{block}{Convex QP}
Consider the convex quadratic program in standard form:
\begin{align*}
  &\min_{x\in\mathbb{R}^n} \; f(x) := \tfrac{1}{2}x^T Q x + c^T x\\
  &\text{subject to }\; A x = b, \quad x \ge 0,
\end{align*}
where $Q\succeq 0$, $A\in\mathbb{R}^{m\times n}$.

First-order (KKT) conditions (existence of multipliers $y,z$):
\begin{align*}
  Qx + c + A^T y - z &= 0 \quad (\text{stationarity})\\
  Ax - b &= 0 \quad (\text{primal feasibility})\\
  X Z e &= 0,\quad x\ge0,\ z\ge0 \quad(\text{complementarity})
\end{align*}
where $X=\mathrm{diag}(x)$, $Z=\mathrm{diag}(z)$, $e$ is the vector of ones.
\end{block}
\begin{block}{Interior-point idea}
   Enforce strict positivity $x>0, z>0$ and drive $XZe=\mu e$ to zero along the \emph{central path} as $\mu\downarrow0$.
\end{block}

\end{frame}

% Slide 2: Barrier method + Newton step + figure
\begin{frame}{A Primer on IPM for convex QP}
  
\only<1>
{
  \begin{block}{Barrier formulation and Newton steps}
  Use a logarithmic barrier to keep $x>0$:
\[
  \phi_\mu(x)=\tfrac{1}{2}x^T Q x + c^T x - \mu \sum_{i=1}^n \log x_i
\]
with equality constraints $Ax=b$. For fixed $\mu>0$ solve
\[\min_{x>0}\; \phi_\mu(x) \quad\text{s.t. } Ax=b.\]
\end{block}
The perturbed KKT conditions for $\mu>0$ are:
\begin{align*}
  Qx + c + A^T y - z &= 0, \\
  Ax - b &= 0, \\
  XZe &= \mu e.
\end{align*}
}
\only<2>{
We linearize these equations at $(x,y,z)$ to get the Newton system for $(\Delta x, \Delta y, \Delta z)$:
\[
  \begin{bmatrix}
    Q & A^T & -I \\
    A & 0   & 0   \\
    Z & 0   & X
  \end{bmatrix}
  \begin{bmatrix}
    \Delta x \\
    \Delta y \\
    \Delta z
  \end{bmatrix}
  = -\begin{bmatrix}
    r_d \\
    r_p \\
    r_c
  \end{bmatrix},
\]
where
\begin{align*}
  r_d &= Qx + c + A^T y - z,\\
  r_p &= Ax - b,\\
  r_c &= XZe - \mu e.
\end{align*}
This 3-block structure is the foundation of primal--dual interior-point methods.
Take a damped Newton step and project back to positive orthant with step-length chosen to maintain $x>0$.
}
% \only<2>
% {
% \vspace{2mm}
% \textbf{Figure: feasible region (linear constraints) and central path for a QP}
% \begin{center}
% \begin{tikzpicture}[scale=1]
%   % feasible polytope (two inequality lines) + ellipse objective contours
%   \filldraw[fill=gray!8] (0.3,0.4) -- (3.2,0.2) -- (3.7,2.8) -- (0.6,2.6) -- cycle;
%   % axes
%   \draw[->] (0,0) -- (4,0) node[right] {$x_1$};
%   \draw[->] (0,0) -- (0,3) node[above] {$x_2$};
%   % objective contours (ellipses)
%   \draw (1.8,1.4) ellipse (1.6 and 0.8);
%   \draw (1.8,1.4) ellipse (1.2 and 0.6);
%   \draw (1.8,1.4) ellipse (0.8 and 0.35);
%   % central path curve
%   \draw[thick,->,red] plot [smooth,domain=0:1] ( {0.5+3*\x^1.2}, {2.5-1.6*\x^0.9} ) node[right]{central path};
%   % current iterates
%   \foreach \t in {0.12,0.28,0.46,0.7,0.9}{\filldraw[black] ({0.5+3*\t^1.2},{2.5-1.6*\t^0.9}) circle (1.8pt);}
%   % optimum
%   \filldraw[blue] (1.8,1.4) circle (2.4pt) node[right] {optimal $x^*$};
% \end{tikzpicture}

% \end{center}
%}
\end{frame}


\begin{frame}{Full Newton step for primal--dual system}
The perturbed KKT conditions for $\mu>0$ are:
\begin{align*}
  Qx + c + A^T y - z &= 0, \\
  Ax - b &= 0, \\
  XZe &= \mu e.
\end{align*}
We linearize these equations at $(x,y,z)$ to get the Newton system for $(\Delta x, \Delta y, \Delta z)$:
\[
  \begin{bmatrix}
    Q & A^T & -I \\
    A & 0   & 0   \\
    Z & 0   & X
  \end{bmatrix}
  \begin{bmatrix}
    \Delta x \\
    \Delta y \\
    \Delta z
  \end{bmatrix}
  = -\begin{bmatrix}
    r_d \\
    r_p \\
    r_c
  \end{bmatrix},
\]
where
\begin{align*}
  r_d &= Qx + c + A^T y - z,\\
  r_p &= Ax - b,\\
  r_c &= XZe - \mu e.
\end{align*}
This 3-block structure is the foundation of primal--dual interior-point methods.
\end{frame}


% Slide 3: Central path illustration
\begin{frame}{Illustration of the central path}
\textbf{Definition:} The \emph{central path} is the trajectory of strictly feasible points $(x(\mu), y(\mu), z(\mu))$ satisfying
\[
Qx(\mu) + c + A^T y(\mu) - z(\mu) = 0, \quad Ax(\mu) = b, \quad X(\mu) Z(\mu)e = \mu e,\; \mu>0.
\]
As $\mu \to 0$, the path converges to the optimal solution $(x^*,y^*,z^*)$.

\vspace{2mm}
\textbf{Geometric intuition:}
\begin{itemize}
  \item Each $\mu$ defines a barrier problem that smooths the feasible region.
  \item The central path traces the minimizers of these smoothed problems.
  \item Interior-point methods follow this path using Newton directions.
\end{itemize}

% \vspace{2mm}
% \begin{center}
% \begin{tikzpicture}[scale=1]
%   % axes
%   \draw[->] (0,0) -- (4.5,0) node[right] {$x_1$};
%   \draw[->] (0,0) -- (0,3.2) node[above] {$x_2$};
%   % feasible region (wedge)
%   \draw[fill=gray!10] (0.5,0.5) -- (4,0.5) -- (4,2.5) -- (0.5,2.5) -- cycle;
%   \node at (3.4,2.7) {feasible region};
%   % objective contours (ellipses)
%   \draw (2,1.4) ellipse (1.6 and 0.8);
%   \draw (2,1.4) ellipse (1.2 and 0.6);
%   \draw (2,1.4) ellipse (0.8 and 0.35);
%   % central path (smooth curve)
%   \draw[thick,red,->] plot [smooth,domain=0:1] ( {0.6+2.8*\x^1.2}, {2.4-1.7*\x^0.9} );
%   % arrows and labels for mu values
%   \foreach \m/\t in {1.0/0.1,0.3/0.4,0.1/0.7,0.03/0.9}{
%     \filldraw[red] ({0.6+2.8*\t^1.2},{2.4-1.7*\t^0.9}) circle (1.5pt);
%     \node[anchor=west,scale=0.8] at ({0.6+2.8*\t^1.2+0.1},{2.4-1.7*\t^0.9}) {$\mu=\m$};}
%   % optimal point
%   \filldraw[blue] (2,1.4) circle (2pt) node[right] {$x^*$};
% \end{tikzpicture}
% \end{center}

\vspace{1mm}
\textbf{Interpretation:} As $\mu$ decreases, the minimizer moves smoothly along the red curve toward the boundary optimum.
\end{frame}

% Slide: Central path in the positive orthant (as in Nocedal & Wright)
\begin{frame}{Central path in the positive orthant (boundary optimum)}
The central path for a problem with $x \ge 0$ approaches the boundary smoothly
as $\mu \to 0$. The iterates remain strictly positive until convergence.

\begin{center}
\begin{tikzpicture}[scale=1.2]
  % Axes
  \draw[->] (0,0) -- (4.2,0) node[right] {$x_1$};
  \draw[->] (0,0) -- (0,3.2) node[above] {$x_2$};
  
  % Feasible region: positive orthant
  \fill[gray!10] (0,0) rectangle (4,3);

  % % Objective contours (ellipses)
  % \foreach \a in {0.6,0.9,1.2,1.5}{
  %   \draw[gray!70] (2.5,1.5) ellipse (1.8/\a and 1.0/\a);
  % }

  % Central path approaching x1=0 boundary
  \draw[thick,red,->,domain=0:1,smooth,variable=\t]
    plot ({0.8*(1-\t)^2 + 0.05},{2.2 - 1.8*(1-\t)^1.3});

  % Points along path (decreasing mu)
  \foreach \m/\t in {1.0/0.0,0.3/0.35,0.1/0.65,0.03/0.9,0.01/0.9}{
    \filldraw[red] ({0.8*(1-\t)^2 + 0.05},{2.2 - 1.8*(1-\t)^1.3}) circle (1.3pt);
  }

  % Labels for mu values
  \node[red!80!black,scale=0.8] at (0.4,2.0) {$\mu\!\to\!0$};
  \node[red!80!black,scale=0.8] at (0.7,1.6) {$\mu=0.1$};
  \node[red!80!black,scale=0.8] at (1.0,1.2) {$\mu=0.3$};
  \node[red!80!black,scale=0.8] at (1.2,0.3) {$\mu=1.0$};

  % Optimal point on the boundary
  
  % \filldraw[blue] (0.0,2.1) circle (2pt);
  % \node[anchor=west,blue,scale=0.9] at (0.0,2.1) {$x^* = (0, x_2^*)$};

  % Annotate the active constraint line
  \draw[dashed,gray!70] (0.05,0) -- (0.05,3);
  \node[gray!70,scale=0.8,rotate=90] at (-0.15,1.5) {active constraint $x_1=0$};

\end{tikzpicture}
\end{center}


\textbf{Interpretation:}
As $\mu \downarrow 0$, $x_1(\mu) \to 0$ and the iterates approach the boundary while maintaining strict positivity, ensuring well-defined Newton steps.
\end{frame}
% Slide 4: Primal-dual IPM, residuals and convergence
\begin{frame}{Primal--dual interior-point method \& practical notes}
Primal--dual methods work with both primal and dual variables and maintain perturbed complementarity:
\[XZ e = \sigma \mu e,\quad \mu=\frac{x^T z}{n},\quad 0<\sigma\le1.
\]
A common predictor--corrector algorithm (Mehrotra) computes a predictor (affine-scaling) step, estimates a \(\sigma\), then corrects to stay close to the central path.

Key practical points:
\begin{itemize}
  \item Use sparse symmetric linear algebra to solve the Newton/KKT system efficiently.
  \item Choose step-lengths to keep iterates positive; use Mehrotra predictor-corrector for fast practical convergence.
  \item Convergence: polynomial worst-case complexity (typically 30--60 iterations for large QPs in practice).
\end{itemize}

Interior-point methods follow the central path using Newton steps on a barrier-regularized KKT system; primal--dual variants maintain both primal and dual feasibility and usually reach high accuracy in few iterations.
\end{frame}





\begin{frame}
  \frametitle{Interior Point Methods for frictional contact}
  \begin{block}{PhD thesis of Hoang Minh Nguyen(2025), with Paul Armand}
   Perturbation of the complementarity condition with a barrier parameter $\tau$\\[2mm]
  
    \begin{columns}[c]
        % \small
      \begin{column}{.5\textwidth}
            \begin{equation*}
              \begin{array}{c}
                \text{ \tr{Original problem}} \\[2mm]
                    M v + f = H^\top r \\
                    H v + w + se = \tilde{u} \\
                    s = \norm{\tilde{u}_\t} \\
                    \tilde{u} \circ r = 0  \\
                    (\tilde{u}, r) \in K^2
                \end{array}
            \end{equation*}
        \end{column}
        \hspace{-1cm}
        \begin{column}{.5\textwidth}
            \begin{equation}
                \label{eq:per-incre-prob}
                \begin{array}{c}
                    \text{ \tr{Perturbed problem}} \\[2mm]
                    M v + f = H^\top r \\
                    H v + w + se = \tilde{u} \\
                    s = \norm{\tilde{u}_\t} \\
                    \tilde{u} \circ r = 2\tau e \\
                    (\tilde{u}, r) \in \mbox{int}(K^2)
                \end{array}
            \end{equation}
        \end{column}
      \end{columns}
    \end{block}
    
    \begin{block}{Convex case ($s$ fixed)}
      IPM is able to solve very accurately and efficiently the problem with a given $s := \| u_\t \|$ even when $H$ is rank-deficient (see \cite{acary:hal-03913568}).\\ Special care of sparse linear systems and conditioning with Nesterov-Todd scaling.
    \end{block}
  \ding{220} Extension to general frictional contact problems: nonsmooth interior point method

  
\end{frame}

\begin{frame}{Nonsmooth Interior-Point Method (NIPM)}
    % \vspace{-0.3cm}

    \tr{Slater's assumption (SA)} \quad $\exists v \in \nbR^m$ \quad such that \quad $Hv+w \in \mbox{int}({K})$
    \begin{theorem}
        \vspace{-5pt}
        \begin{enumerate}
            \item Under SA, for each $\tau > 0$, the perturbed problem \eqref{eq:per-incre-prob} has a solution $(v_\tau,\tilde{u}_\tau,r_\tau,s_\tau)$ \\[6pt]
            \item Under SA, there exists an analytic central path $\{(v_\tau,\tilde{u}_\tau,r_\tau,s_\tau): \tau>0\}$, which converges to a solution of the original problem
        \end{enumerate}
      \end{theorem}
      Proof: Fixed point  Brouwer's Theorem and curve selection lemma in semi-algebraic analysis.
      
      \begin{block}{Main theoretical outcome}
        \begin{itemize}
        \item Alternative proof of solution existence for $\mathrm{FC/I}(M,H,f,w,\mu)$
        \item  The central path is not necessarily unique !
        \end{itemize}
      \end{block}
\end{frame}


\begin{frame}{Nonsmooth Interior-Point Method (NIPM) - Linearization}
  \begin{block}{{Iterations and Jacobian matrix }}
    \begin{columns}[c]
      \begin{column}{.5\textwidth}
        \begin{equation*}
          G:=\begin{bmatrix} Mv+f-H^\top r \\ Hv+w-\tilde{u}+se \\ s - \norm{\tilde{u}_\t} \\ \tilde{u} \circ r \end{bmatrix}
          = \begin{bmatrix} 0 \\ 0 \\ 0 \\ 2\tau e \end{bmatrix}
        \end{equation*}
        \quad
      \end{column}
      \begin{column}{.5\textwidth}
        \begin{equation*}
          J:=\begin{bmatrix} M & -H^\top & 0 & 0 \\ H & 0 & -I & e \\ 0 & 0 & -L & 1 \\ 0 & \tilde{U} & R & 0 \end{bmatrix}
        \end{equation*}
        \quad
      \end{column}
    \end{columns}
    % \vspace{0.1cm}
     where $L = \begin{pmatrix} 0 & \partial \norm{\tilde{u}_\t}^\top \end{pmatrix}$, \quad $\mbox{with } \partial \norm{\tilde{u}_\t} = \begin{cases} \frac{\tilde{u}_\t}{\norm{\tilde{u}_\t}} & \mbox{if } \tilde{u}_\t \neq 0 \\ d \in \mathbb B & \mbox{if } \tilde{u}_\t = 0  \end{cases}$ (unit ball $\mathbb B$)
  \end{block}
    %
  \begin{block}{Linear system and Mehrotra's-Like algorithm}
    $$J \, d = -G + \left[ \begin{matrix} 0 \\ 0 \\ 0 \\ 2\sigma \tau e \end{matrix}\right]$$ predictor-corrector scheme $\sigma \in (0,0.5)${\small: centralization parameter}
  \end{block}
   \begin{block}{Stopping test}
     $$\max\left\{ \; \|Hv+w-\tilde{u}\|_\infty, \; \|Mv+f-H^\top r\|_\infty, |s - \norm{\tilde{u}_\t}|, \; \|\tilde{u} \circ r\| \; \right\}  \; \leq \; \mbox{tol}$$
   \end{block}
 \end{frame}

 \begin{frame}
   \frametitle{Nonsmooth Interior-Point Method (NIPM)}
   
   \begin{algorithm}[H]
     \scriptsize
  \caption{Mehrotra type Nonsmooth Interior Point Method (NIPM)}
  \label{alg:IP}
  %\hspace*{\algorithmicindent} 
  \textbf{Parameters:} Choose a starting point $(v,r,u,s)$ such that $(r,u)\in{\rm int}({\cal K}^2)$. Choose $\eta_1\in(0,1)$, $\eta_2\geq 1$, $\eta_3\geq 1$, $\gamma_1\in(0,1)$, $\gamma_2\in (0,1-\gamma_1)$, $c_1\geq 1$, $c_2\in(0,1)$, $\rm{tol}>0$ and set $\gamma = 0.99$;
  \begin{algorithmic}[1]
    \State\label{step1}{\bf if} the stopping criterion is satisfied {\bf then} return $(v,r,u,s)$ as solution of \eqref{eq:math-model}; 
    \State\label{step2} Set $\tau\leftarrow u^\top r / n$;
    \State\label{step3} {\bf if} $\|s-\ell(u)\|_2 \geq c_1 \| u \circ r\|_2$ {\bf then} set $\sigma \leftarrow 0.998$, $d^u_a \circ d^r_a \leftarrow 0$ and go to \ref{step10},
    \State\label{step4} {\bf else} compute $d_a=(d^v_a,d^r_a,d^u_a,d^s_a)$ solution of
    \begin{displaymath}
      J(v,r,u,s) d_a = -G(v,r,u,s);
    \end{displaymath}
    \State\label{step5} Find the greatest $\alpha_a\in(0,1]$ such that $(r,u)+\alpha_a(d_a^r,d_a^u) \in {\cal K}^2$;
    \State\label{step6} Set $\tau_a \leftarrow  (u+\alpha_a d^u_a)^\top(r+\alpha_a d^r_a)/n$;
    \State\label{step7} {\bf if} {$\tau>\eta_1$}\,  {\bf then} set $\beta\leftarrow \max\{1,\eta_2 \alpha_a^2\}$\, {\bf else} set $\beta \leftarrow \eta_3$;
    \State\label{step8} Set $\sigma \leftarrow 0.998\min\{1,(\tau_a/\tau)^\beta\}$;
    \State\label{step9} {\bf if}  $\sigma \geq c_2$  {\bf then} set $d^u_a \circ d^r_a \leftarrow 0$;
    \State\label{step10} Compute $d=(d^v,d^r,d^u,d^s)$ solution of
    \begin{equation}
      \label{eq:JdG}
      J(v,r,u,s) d = -G(v,r,u,s)+\left[\begin{smallmatrix} 0\\0\\-d^u_a \circ d^r_a + \sigma\tau e\\ 0\end{smallmatrix}\right]
    \end{equation}
    \State\label{step11} Find the greatest $\alpha\in(0,1]$ such that $(r,u)+\alpha(d^r,d^u) \in (1-\gamma)(r,u) + {\cal K}^{2}$;
    \State\label{step12} Set $\gamma \leftarrow \gamma_1 + \alpha \gamma_2$;
    \State\label{step13} Set $(v,r,u,s) \leftarrow (v,r,u,s) + \alpha (d^v, d^r,d^u,d^s)$ and goto \ref{step1}.
   \end{algorithmic}
 \end{algorithm}

 \end{frame}


\begin{frame}
  \frametitle{Nonsmooth Interior-Point Method (NIPM)}
   \begin{center}
    \includegraphics[height=0.7\textheight]{./figure/IPM/solver_performance_11_100.jpg}
  \end{center}
  \begin{block}{Problem's size $11\leq n \leq 100$}
    IPM (GFC3D) outperforms NSGS and ADMM in terms of efficiency and robustness
  \end{block}
\end{frame}
\begin{frame}
  \frametitle{Nonsmooth Interior-Point Method (NIPM)}
  
  \begin{center}
    \includegraphics[height=0.7\textheight]{./figure/IPM/solver_performance_1000_.jpg}
  \end{center}

  \begin{block}{Problem's size $1000\leq n $}
    IPM (GFC3D) suffers from robustness
  \end{block}
  
\end{frame}

\begin{frame}{Nonsmooth Interior-Point Method (NIPM) - failures}
    \vspace{-0.1cm}
    % \begin{exampleblock}{Questions}
    %     {\color{greenexampleblock}$\bullet$} Why we require such high accuracy, specifically up to $10^{-14}$ ? \\
    %     $\to$ To ensure an accurate identification of contact status \\[6pt]
    %     {\color{greenexampleblock}$\bullet$} What is the cause of failures ?
    % \end{exampleblock}
    \begin{exampleblock}{Failure \#1: {\color{black} A special shape of the central path}}
        \begin{center}
            \includegraphics[width=0.495\linewidth]{./figure/IPM/images/ncv_fail1a.png}
            \includegraphics[width=0.495\linewidth]{./figure/IPM/images/ncv_fail1b.png}
        \end{center}
        \begin{center}
        Solution: \quad {\color{blue} $r$}$^\times${\color{blue} $\in \mbox{int}({K})$}, \quad {\color{red} $\tilde{u}$}$^*${\color{red} $= 0$} \quad (sticking)
        \end{center}
    \end{exampleblock}
    This shape of the central path can cause iterates to get stuck on the boundary, which is not the correct position for the solution.
\end{frame}




\begin{frame}{Nonsmooth Interior-Point Method (NIPM) - failures}
    \vspace{-0.1cm}
    {\large Failure \#2:} Non-monotone parameterization of the central path \\[4pt]
    {$\bullet$} {\small {\color{red}Red}-{\color{blue}blue} curve: Central path $\tau \to r(\tau)$ calculated by Asymptotic Numerical Method (ANM). {\color{red}Red}: $\tau$ decreases. {\color{blue}Blue}: $\tau$ increases \\[6pt]
    % ===============================================================
    {$\bullet$} \textbf{Black} curve: the path of NIPM iterates}
    \begin{center}
        \includegraphics[width=0.32\linewidth]{./figure/IPM/images/anm_cone6.png}
        \includegraphics[width=0.32\linewidth]{./figure/IPM/images/anm_cone6_2.png}
        \includegraphics[width=0.34\linewidth]{./figure/IPM/images/tau2t.png}
    \end{center}
    \tr{\ding{220} \large Asymptotic Numerical Method (ANM):} algorithm based on the computation of series to perform accurate numerical continuations of parameterized  non-linear problems

  \end{frame}
  \begin{frame}{Nonsmooth Interior-Point Method (NIPM) and ANM}
    \begin{block}{Reformulation as bilinear problem}
      Let us write the perturbed problem ~\eqref{eq:per-incre-prob} under the form
      \begin{equation}
        \label{eq:F}
        F(x,\tau) = G(x)-\tau \begin{pmatrix} 0\\0\\2e \\0\end{pmatrix} = 0,
      \end{equation}
      where $x=(v,r,u,s)$ and takes advantage of the fact that $G(x)$ can be written as a sum of linear term, a constant term and bilinear function as
\begin{equation}
  \label{eq:1000000}
  L(x) + b + Q(x,x) 
\end{equation}
with
\begin{equation}
  \label{eq:18}
  x =
  \begin{pmatrix}
   v \\ u \\ r \\ s 
  \end{pmatrix},\quad
   L =
   \begin{pmatrix}
     M &  -H^\top &   & 0\\
     -H & 0 & I & -E \\
     0 & 0 & 0 & 0 \\
     0 & 0 & 0 & 0\\
  \end{pmatrix},\quad
  Q(x,\tilde x) =
  \begin{pmatrix}
    0 \\
    0 \\
    u \circ \tilde r\\
    s\bullet  \tilde s - l(u) \bullet  l(\tilde u )\\
  \end{pmatrix}
\end{equation}
and
\begin{equation}
  F =
  \begin{pmatrix}
    0 &
    0 &
    2 e &
    0
  \end{pmatrix}^\top,\quad
  b =
  \begin{pmatrix}
    - f&
    - w &
    0 &
    0
  \end{pmatrix}^\top,\quad \lambda= \tau.
\end{equation}
\end{block}
\end{frame}


\begin{frame}{Nonsmooth Interior-Point Method (NIPM) and ANM}
 \begin{block}{Principles of Asymptotic Numerical method (ANM)}
  
      The principle of ANM is to start from an initial known vector $(x_0,\tau_0)$ such that $F(x_0,\tau_0)=0$, then to calculate a solution of \eqref{eq:F} under the form of truncated series
      \begin{displaymath}
        x(t) = \sum_{k=0}^N x_k a^k \quad \tau(t) = \sum_{k=0}^N \tau_k a^k,
      \end{displaymath}
    
      \ding{220} In our particular case, bilinear function $\implies$ constant Jacobian for a computation of all the terms of the series.
      \begin{equation}
        \label{eq:23}
        J(x) =
        \begin{pmatrix}
          M &  -H^\top &   & 0\\
          -H & 0 & I & -E \\
          0 & U & R & 0 \\
          0 & 0 & -A & 2\, {\rm diag}(s) \\
        \end{pmatrix}
      \end{equation}
      with
      \begin{equation}
        \label{eq:22}
        A =2\, {\rm diag}(
        \begin{bmatrix} 0 & u_\alpha^\top
        \end{bmatrix}
        , \alpha \in \{1\ldots m\}]).
      \end{equation}
  \end{block}
\end{frame}

\begin{frame}{Nonsmooth Interior-Point Method (NIPM) and ANM}
 \begin{block}{Principles of Asymptotic Numerical method (ANM)}
  \textbf{Zeroth order ($x_0 a^0$).}   For $k=0$, the substitution of (\ref{eq:ANM_3}) in (\ref{eq:ANM_R}) gives
\begin{equation}
  \label{eq:4}
  R(x_0,\lambda_0) = L(x_0) + b + Q(x_0,x_0) -\lambda_0 F  =0
\end{equation}
\textbf{First order ($x_1 a^1$).}
\begin{equation}
  \label{eq:6}
  \begin{cases}
  J(\bar x_1) =  F \\
  \lambda^2_1 = \frac{s^2}{1+\|\bar x_1\|^2}\\
    x_1 = \lambda_1 \bar x_1
  \end{cases}
\end{equation}
\textbf{Higher orders ($x_k a^k$, $k \geq 2$).}  
\begin{equation}
  \label{eq:15}
  \left\{
    \begin{array}{l}
      J \bar x_k =- \sum_{i=1}^{k-1}   Q(x_i,x_{k-i})\\[2mm]
      \lambda_k = - \frac{\lambda_1 x_1^\top \bar x_k}{s^2}\\[2mm]
      x_k  = \frac{\lambda_k}{\lambda_1} x_1+ \bar x_k.
  \end{array}\right.
\end{equation}




 \end{block}
\end{frame}
\begin{frame}{Nonsmooth Interior-Point Method (NIPM) and ANM}
 \begin{block}{Principles of Asymptotic Numerical method (ANM)}
\textbf{Radius of convergence}
The radius of convergence for a user tolerance $\rm tol$ is approximated as follows:
\begin{equation}
  \label{eq:16}
  a_{\max} = \left({\rm tol}\frac{\|x_1\|}{\|x_N\|} \right)^{1/(N-1)}
\end{equation}
ANM is able to calculate the central path with very tight tolerance ($\leq 10^{-14}$)
\end{block}
\end{frame}

\begin{frame}{Nonsmooth Interior-Point Method (NIPM) and ANM}
\begin{figure}[htbp]
  \begin{center}
    \includegraphics[width=0.9\linewidth]{ipm-anm-convergence-primitive_vs_a.pdf}
\caption{Values of $\tau_k$ w.r.t $a$ for IPM-ANM for solving a problem with eleven contact points with accuracy $10^{-14}$. Each marker indicates a new computation of an asymptotic sum. Bottom graph: a zoom on the last $4$ ANM-IPM iterations.}
\label{tau2t_vs_a}
\end{center}
\end{figure}
\end{frame}
\begin{frame}{Nonsmooth Interior-Point Method (NIPM) and ANM}
  \begin{figure}[htbp]
  \begin{center}
    \includegraphics[width=0.45\linewidth]{./figure/anm_cone6.png}
    \includegraphics[width=0.45\linewidth]{./figure/anm_cone6_2.png}
\caption{Behavior of Algorithm~\ref{alg:IP} when solving PrimitiveSoup-ndof-6000-nc-1087-183. The projection of the central path in the cone 6 is represented by the red and blue curve with small circles. The dotted blue part of the curve corresponds to ANM iterations for which $\tau$ increases. The black curve represents the path of the interior point iterates. We can see that as $\tau$ decreases, the iterates of Algorithm~\ref{alg:IP} follow the central trajectory, but that as $\tau$ increases, the algorithm is no longer able to generate correct directions to continue following this trajectory.}
\label{anm_cone6}
\end{center}
\end{figure}
\end{frame}
\begin{frame}{Nonsmooth Interior-Point Method (NIPM) and ANM}
\begin{block}{Takeway}
  The central path is not monotonically parameterized and sometimes slides on the boundary of feasible domains.
  \ding{220} Need for a precise and fully functional continuation method: ANM
\end{block}
\end{frame}



\begin{frame}
  \frametitle{Nonsmooth Interior-Point Method (NIPM)}
  \begin{block}{Moderate size problems}
    IPM with ANM is robust (but slow for large problems)
  \end{block}
  \def\figheightanm{0.42}
  
  \begin{center}
    \includegraphics[height=\figheightanm\textheight]{./figure/IPM/anm_solver_performance_1_10.png}
    \includegraphics[height=\figheightanm\textheight]{./figure/IPM/anm_solver_performance_11_100.png}
    \includegraphics[height=\figheightanm\textheight]{./figure/IPM/anm_solver_performance_101_1000.png}
    \includegraphics[height=\figheightanm\textheight]{./figure/IPM/anm_solver_performance_1000_.png}
 \end{center}
\end{frame}



%%% Local Variables:
%%% mode: latex
%%% TeX-master: "s"
%%% End:






\section{Conclusions \& Perspectives}

\frame{
  \frametitle{Conclusions \& Perspectives}
  \begin{block}{Conclusions}
    \begin{itemize}
    \item Further research is still needed for an robust AND efficient solver.
    \item IPM and ANM numerical method provides a robust solver.
    \item Coupling with other physical phenomena to obtain a monolithic variational inequality : \\[1mm]
      $\bullet$ (non associated) plasticity \citep{acary:hal-03978387,guillet:hal-05070887}\\
      $\bullet$ fracture with cohesive zone model \citep{collinscraft:hal-03371667} \\
      $\bullet$ damage mechanics.
    \end{itemize}
  \end{block}


  
  \begin{block}{Open software and data collections.}
    \begin{itemize}
    \item Siconos/Numerics. A open source collection of solvers. \\
      {\url{https://github.com/siconos/siconos}}  
    \item FCLIB: a open collection of discrete 3D Frictional Contact (FC) problems \\
      {\href{https://github.com/FrictionalContactLibrary}{https://github.com/FrictionalContactLibrary}}  contribute ...
    \end{itemize}
    Use and contribute ...
  \end{block}
  % \begin{block}{Perspectives}
  %   \begin{itemize}
  %   \item Nonlinear discretized equations (dynamics or quasi-statics)\\
  %     finite strains, finite rotations, hyperelastic models, \ldots
  %   \item Plasticity and damage, cohesive zone element coupled with contact and friction\\
  %     formulation as a monolithic variational inequality 
  %   \end{itemize}
  %\end{block}
}


% \frame{
% \frametitle{Conclusions \& Perspectives}

% \only<1>
% {\begin{block}{Conclusions}
%   \begin{enumerate}
%   \item A bunch of articles in the literature \\
%     47000 articles since 2000 on ``Coulomb friction numerical method'' in Google Scholar.
%   \item No ``Swiss--knife'' solution : choose efficiency OR robustness  
%   \item Newton--based solvers solve efficiently some problems, but robustness issues
%   \item First order iterative methods ($\sf VI, NSGS, PSOR$) solves all the problems but very slowly  
%   \item The rank of the $H$ matrix ($\approx$ratio number of contacts unknows/number of d.o.f) plays an important role on the robustness
%   \item Optimisation-based  and proximal-point algorithm solvers are  interesting but it is difficult to forecast theirs efficiencies.
%   \item Need for a second order method when $H$ is rank-deficient (IPM?)
%   \end{enumerate}
% \end{block}
% }
% \only<2>
% {
% \begin{block}{Perspectives}
%   \begin{enumerate}
%   \item Develop new algorithm and compare other algorithm in the literature. \\
%     (interior point techniques, issues with standard optimization software.)
%   \item Improve the robustness of Newton solvers and accelerate first-order method
%   \item Complete the collection of benchmarks  \ding{220} FCLIB
%   \end{enumerate}
% \end{block}
% }
% }


\frame
{

  % All the results may be found in \cite{Acary.ea_Chapter2018}\\[1mm]
  
  % {\em On solving frictional contact problems: formulations and comparisons of numerical methods. Acary, Br\'emond, Huber.
  %   Advanced {T}opics in {N}onsmooth {D}ynamics, Acary, V. and Br\"uls. O. and Leine, R. (eds). Springer Verlag. 2018}

  \vspace{1cm}
  \centerline{\textcolor{red}{ Thank you for your attention.}}

  \begin{center}
    Thanks to the collaborators for stimulating discussions and developments:\\[2mm]
    
    Pierre Alart, Paul Armand, Florent Cadoux, Fr\'ed\'eric Dubois,\\
    Claude L\'emar\'echal, J\'er\^ome Malick and Mathieu Renouf 
  \end{center}
}
\begin{frame}
\begin{center}
    \large{Journal of Theoretical Computational and Applied Mechanics}\\[2mm]
    
    \includegraphics[scale=0.75]{Logo.pdf}\\[3mm]

    {The only overlay Diamond Open Access journal in Mechanics}\\[1mm]

    \large\textcolor{red}{Right now: an open call for associate editors}


\end{center}
\vspace{1mm}
\begin{itemize}
    \item Diamond Open Access: free for readers and authors
    \item Overlay: based on open archives (arXiv, Hal, ...)
    \item Publications of the highest scientific calibre with open reviews
    \item A community--supported journal with ethical and  FAIR principles
    \item Promotion of reproducible and open science
\end{itemize}
\vspace{1mm}
\begin{center}

% \begin{minipage}{0.35\textwidth}
%   \includegraphics[width=0.5\textwidth]{fist.jpg}
% \end{minipage}
%\begin{minipage}{0.45\textwidth}
  \large\textcolor{red}{Get involved. Join us now. Contributions are welcome!}\\
  
  
%\end{minipage}

\end{center}
\end{frame}


 
\def\newblock{}
{\scriptsize
\printbibliography
}

\end{document}

%%% Local Variables: 
%%% mode: latex
%%% TeX-master: t
%%% End: 
