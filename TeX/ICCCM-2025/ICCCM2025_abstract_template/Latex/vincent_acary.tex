\documentclass[10pt]{icccm2025}
\usepackage{epsfig,amsfonts, amsmath}

\usepackage{hyperref}
\hypersetup{
    colorlinks=true,
    linkcolor=blue,
    filecolor=magenta,      
    urlcolor=cyan,
    pdftitle={Overleaf Example},
    pdfpagemode=FullScreen,
    }


\newcommand{\RR}{\ensuremath{\rm I\!R}}
\newcommand{\NN}{\ensuremath{\rm I\!N}}
\def\n{{\hbox{\tiny{N}}}}
\def\t{{\hbox{\tiny{T}}}}

\usepackage{graphicx,subfig}
\usepackage[numbers,square,sort&compress]{natbib}
\bibliographystyle{unsrt}
 
\title{Numerical solutions of the Coulomb friction contact problem from the perspective of optimisation and mathematical programming.}

\author{\underline{Vincent Acary}$^{1}$}

\bigskip
\address{{\em $^{1}$ INRIA, Grenoble Alpes University center, Grenoble, France.
\\ E-mail: vincent.acary@inria.fr \\
}}


\keywords{Coulomb's friction,  optimization, variational inequalities, complementarity problems.}


\begin{document}
\vspace{-0.8cm}
\def\figillus{0.1\textheight}
\begin{figure}[htbp]
  \centering
  \subfloat[Cubes\_H8]{\includegraphics[height=\figillus]{figure/Cubes_H8_5.png}}
  \subfloat[LowWall\_FEM]{\includegraphics[height=\figillus]{figure/LowWall_FEM.png}}
  \subfloat[Aqueduct\_PR]{\includegraphics[height=\figillus]{figure/Aqueduc_PR.png}}
  %\subfloat[Bridge\_PR]{\includegraphics[height=\figillus]{figure/Bridge_PR_1.png}}
  %\subfloat[100\_PR\_Periobox]{\includegraphics[height=\figillus]{figure/100_PR_PerioBox.png}}
  %\subfloat[945\_SP\_Box\_PL]{\includegraphics[height=\figillus]{figure/945_SP_Box_PL.png}}
  % \subfloat[Capsules]{\includegraphics[height=\figillus]{figure/Capsules.png}$\quad$}
  %\subfloat[Chain]{\includegraphics[height=\figillus]{figure/Chains.png}$\quad$}
  %\subfloat[KaplasTower]{\includegraphics[height=\figillus]{figure/KaplasTower.png}$\quad$} \subfloat[BoxesStack]{$\quad$$\quad$\includegraphics[height=\figillus]{figure/BoxesStack.png}$\quad$$\quad$}\\
\subfloat[Chute\_4000]{\includegraphics[height=\figillus]{figure/Chute_1000_light.jpg}}
  \caption{Illustrations of the FClib test problems}
  \label{fig:fclib}
\end{figure}
\vspace{-0.3cm}
\noindent In this presentation, we propose to present numerical methods for solving contact with Coulomb friction from the perspective of methods derived from optimisation or mathematical programming.
The aim is to review effective and recognised methods and the challenges they still pose for the community.

 The numerical solution of contact problems with Coulomb friction is known to be challenging.
While the problem without friction can be reduced to a convex optimisation problem, at least for the case of small perturbations in linear elasticity or associated plasticity, Coulomb friction does not correspond to a  convex optimisation problem.
This is mainly due to the fact that sliding occurs without dilation and the principle of maximum dissipation is only true for the tangential component of the contact.
%Another way of looking at this difficulty is that sliding occurs without dilation, so the model does not derive from a pseudo convex dissipation potential.

This has important consequences for the numerical methods, as we can no longer make direct use of the tried and tested methods of convex optimisation.
%We must either design new ones, adapt optimisation methods or use work on variational inequalities and complementarity problems.
The computational contact mechanics community has proposed numerous methods for addressing these challenges, adapting existing optimisation or variational inequalities-based approaches~\cite{Acary.Brogliato2008,Wriggers2006,Laursen2003,Dostal.ea_Book2023}.
Notably, the community has been a trailblazer in this field, pioneering methods such as the non-smooth Newton methods~\cite{Alart.Curnier1991}.

\vspace{-0.5cm}
\paragraph{Statement of the problem} To concentrate on the issues associated with solving the frictional contact, we will consider a linearised discrete problem, resulting for example from a time discretisation, or from an incremental formulation after a space discretisation and possible linearization~\cite{Acary.Cadoux2013}.




Given a symmetric positive definite matrix ${M} \in \RR^{n \times n}$, a vector $ {f} \in \RR^n$, a matrix  ${H} \in \RR^{n \times m}$ with $m= 3n_c$ where $n_c$ is the number of contact points, a vector $w \in \RR^{m}$ and a vector of coefficients of friction $\mu \in \RR^{n_c}$ , the discrete frictional contact problem is to find three vectors $ {v} \in \RR^n$, $u\in\RR^m$ and $r\in \RR^m$ such that
\begin{equation}\label{eq:soccp1}
  \begin{array}{c}
    M v = {H} {r} + {f}, 
    u = H^\top v + w,  
                         \hat u = u + g(u), \quad
    g(u) = \left[
\begin{bmatrix}
\mu^\alpha  \|u^\alpha_\t\| \\ 0
\end{bmatrix}
, \alpha = 1\ldots n_c\right]^\top \\[1ex]
                        K^\star \ni {\hat u} \perp r \in K,
  \end{array}
\end{equation}
For each contact~$\alpha$, the unknown variables  $u^\alpha\in\RR^3$ (velocity or gap at the contact point) and $r^\alpha\in\RR^3$ (reaction or impulse) are decomposed  in a contact local frame  such that $u^\alpha =
\begin{pmatrix}
  u^\alpha_{\n}, u^\alpha_{\t}
\end{pmatrix},
u^\alpha_{\n} \in \RR, u^\alpha_{\t} \in \RR^2$ and  $r^\alpha = \begin{pmatrix}
  r^\alpha_{\n}, r^\alpha_{\t}
\end{pmatrix}, r^\alpha_{\n} \in \RR, r^\alpha_{\t} \in \RR^2$.
%The cone and $K\subset \RR^{3 n_c}$ is a Cartesian product of second order cone in $\RR^3$.  
% The Coulomb friction cone for a  contact $\alpha$ is defined by $K^{\alpha}  = \{r^\alpha, \|r^\alpha_\t \| \leq \mu^\alpha |r^\alpha_\n| \}$ and the set $K^{\alpha,\star}$ is its dual.  
The cone $K$ is the cartesian product of Coulomb's friction cone at each contact, that is
\begin{equation}
  \label{eq:CC}
  K = \prod_{\alpha=1\ldots n_c} K^{\alpha}  = \prod_{\alpha=1\ldots n_c} \{r^\alpha, \|r^\alpha_\t \| \leq \mu^\alpha |r^\alpha_\n| \},
\end{equation}
and $K^\star$ is dual. For more details, we refer to ~\cite{acary2018solving}.
In~\cite{Acary.ea_ZAMM2011}, a existence result is given for this problem providing a feasibility assumption of Slater type if satisfied. It is interesting to note that this simple condition can be tested numerically. From the point of view of mathematical programming, problem~\eqref{eq:soccp1} is a Second-Order Cone Complementarity Problem(SOCCP).
% If the nonsmooth part of the problem is neglected ($g(u)=0$), the problem is an associated friction problem with dilatation, which is incidentally a mild SOCLCP with a positive matrix $H^\top M^{-1} H$  (possibly semidefinite).
If the non-associated nature of the friction is taken into account by $g(u)$, the problem is non-monotone and non-smooth, and then very difficult to solve efficiently, especially in the case of hyperstatic rigid multi-body systems when $H$ is not full rank.

\vspace{-0.3cm}
\paragraph{Comparison of numerical methods}We will review  several  classes of existing algorithms for solving this problem:
\vspace{-0.2cm}
\begin{itemize}
  \setlength{\parsep}{-4pt}
  \setlength{\itemsep}{-4pt}
  %\setlength\itemsep{0em}
\item variational inequalities solvers: fixed point and extragradient  methods,
\item nonsmooth equations solvers: semi--smooth and generalized Newton methods with line-searches,
\item block--splitting (Gauss-Seidel like) and projected overrelaxation (PSOR),
\item proximal point algorithms and ADMM solvers,
\item optimization based solvers: Panagatiopolous approach, Czech school approach (Tresca successive approximations) and convex SOCQP relaxation \cite{Dostal.ea_Book2023},
\item interior point methods~\cite{acary:hal-03913568}.
\end{itemize}
\vspace{-0.2cm}
The goal is to compare them, on a large set of problems, using the software \href{http://www.siconos.org}{\sc Siconos} and to propose some new approaches. To this end, we build an open collection of discrete frictional contact problems called~\href{https://frictionalcontactlibrary.github.io/index.html}{FCLIB} in order to offer a large library of problems to compare algorithms on a fair basis. 

%%%%%%%%%%%%%%%%%%%%%%%%%%%%%%%%%%%%%%%%%%%%%%%%%%%%%%%%%%%%

\vspace{-1.5ex}
\small
%\bibliography{biblio}
\begin{thebibliography}{1}
\bibitem{Acary.Brogliato2008}
V.~Acary and B.~Brogliato.
\newblock {\em {Numerical methods for nonsmooth dynamical systems. Applications
  in mechanics and electronics.}}
\newblock {Lecture Notes in Applied and Computational Mechanics 35. Berlin:
  Springer. xxi, 525~p. }, 2008.
\vspace{-1.5ex}
\bibitem{Wriggers2006}
P.~Wriggers.
\newblock {\em Computational Contact Mechanics}.
\newblock Springer Verlag, second edition, 2006.
\newblock originally published by John Wiley \& Sons Ltd., 2002.
\vspace{-1.5ex}
\bibitem{Laursen2003}
T.A. Laursen.
\newblock {\em Computational Contact and Impact Mechanics -- Fundamentals of
  Modeling Interfacial Phenomena in Nonlinear Finite Element Analysis}.
\newblock Springer Verlag, 2003.
\newblock 1st ed. 2002. Corr. 2nd printing,.
\vspace{-1.5ex}
\bibitem{Dostal.ea_Book2023}
Zdeněk Dostál, Tomáš Kozubek, Marie Sadowská, and Vít Vondrák.
\newblock {\em Scalable Algorithms for Contact Problems}.
\newblock Springer International Publishing, 2023.
\vspace{-1.5ex}
\bibitem{Alart.Curnier1991}
P.~Alart and A.~Curnier.
\newblock A mixed formulation for frictional contact problems prone to
  \uppercase{N}ewton like solution method.
\newblock {\em Computer Methods in Applied Mechanics and Engineering},
  92(3):353--375, 1991.
\vspace{-1.5ex}
\bibitem{Acary.Cadoux2013}
V.~Acary and F.~Cadoux.
\newblock {\em Recent Advances in Contact Mechanics, Stavroulakis, Georgios E.
  (Ed.)}, volume~56 of {\em Lecture Notes in Applied and Computational
  Mechanics}, chapter Applications of an existence result for the {C}oulomb
  friction problem.
\newblock Springer Verlag, 2013.
\vspace{-1.5ex}
\bibitem{acary2018solving}
Vincent Acary, Maurice Br{\'e}mond, and Olivier Huber.
\newblock On solving contact problems with coulomb friction: formulations and
  numerical comparisons.
\newblock In {\em Advanced Topics in Nonsmooth Dynamics: Transactions of the
  European Network for Nonsmooth Dynamics}, pages 375--457. Springer, 2018.
\vspace{-1.5ex}
\bibitem{Acary.ea_ZAMM2011}
V.~Acary, F.~Cadoux, C.~Lemar\'echal, and J.~Malick.
\newblock A formulation of the linear discrete coulomb friction problem via
  convex optimization.
\newblock {\em ZAMM - Journal of Applied Mathematics and Mechanics /
  Zeitschrift f\"ur Angewandte Mathematik und Mechanik}, 91(2):155--175, 2011.
\bibitem{acary:hal-03913568}
  \vspace{-1.5ex}
Vincent Acary, Paul Armand, Hoang Minh~Nguyen, and Maksym Shpakovych.
\newblock {Second order cone programming for frictional contact mechanics using
  interior point algorithm}.
\newblock {\em {Optimization Methods and Software}}, 39(3):634--663, 2024.
\end{thebibliography}


\end{document}



%%% Local Variables:
%%% mode: latex
%%% TeX-master: t
%%% End:
