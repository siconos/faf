\documentclass[a4paper]{article}

\usepackage{listing}
\usepackage{caption}
\usepackage{minted}
\usepackage{RR}
\usepackage{a4wide} 

\RRdate{\today}
\RRauthor{Maurice Br\'emond {\tt
    maurice.bremond@inria.fr}}
\authorhead{}
\RRtitle{A code generation
  procedure with verifications for the computation of generalized jacobians}
\RRetitle{}
\titlehead{}
\RRresume{}
\RRabstract{Given the expression of a nonsmooth
  function in Sympy, a Python computer algebra system, an element of its
  generalized jacobian may be expressed as a piecewise function. This
  piecewise function takes its values on the smooth domain, where it is the
  smooth jacobian computed by the computer algebra system, and on the
  nonsmooth domain where some values must be provided explicitely. With the
  help of a small tool a C code is generated from this piecewise function. If
  the expression of the generalized jacobian is correct, for inputs in a
  bounded domain, the computation of the jacobian must gives finites
  results. This may be proven with the help of added ACSL static assertions to
  give hints to the frama-c value analysis tools and its subsequents proofs
  obligations.}
\RRmotcle{}
\RRkeyword{}
\RRprojet{Bipop}
\URRhoneAlpes
\RCGrenoble

\PassOptionsToPackage{dvipsnames,svgnames}{xcolor}   
\renewcommand{\baselinestretch}{1.5}
\usepackage{natbib}


\usepackage{geometry}
\usepackage{layout}
\geometry{
  left=2.5cm,
  right=2.5cm,
  top=3.5cm,
  bottom=3cm
}

\usepackage{natbib}
% For unicode 
%\usepackage{ucs}
\usepackage[utf8]{inputenc}
%\usepackage[utf8]{inputenc}
\usepackage[T1]{fontenc}
\usepackage[english]{babel}

% invisible unicode space
\DeclareUnicodeCharacter{00A0}{~}

\usepackage{mathtools, amsthm, amssymb}

\usepackage{graphicx}
\usepackage{psfrag}

%\usepackage{subfigure}
\usepackage{subfig}
\usepackage{float}
\usepackage{rotating}

% table package
\usepackage{hhline}
\usepackage{multicol} % for column
\usepackage{tabularx} % for beautiful column

% math packages 
\usepackage{amsmath,amsthm,amsfonts,amssymb,amsbsy,stmaryrd} 

\usepackage{subeqnarray}



% pour la cesure des mots
%\usepackage[T1]{fontenc}
%\usepackage{inputenc}
\usepackage{textcomp}

%pour l'environnement remarque
%\usepackage{theorem}
%\usepackage{ntheorem}

%pour faire des symboles d\'ebiles
\usepackage{pifont}
%pour faire des minis tables des matieres

\usepackage{algorithm}
\usepackage{algorithmic}

%\usepackage[numbers]{natbib}

\renewcommand{\cite}[1]{\protect\citep{#1}}
%\def\cite#1{\citep{#1}}
%\usepackage{draftcopy}

\usepackage{float}
%\usepackage{showkeys}
%\usepackage{showlabels}

\usepackage{array}



\usepackage{tikz}
\usepackage{xkeyval,tkz-base}
\usetikzlibrary{arrows}
\usetikzlibrary{calc}

\usepackage{fancyhdr}
\usepackage{lastpage}


\usepackage[pdfencoding=unicode,pdftex,backref=page]{hyperref}


\usepackage{pdflscape}
\usepackage[skip=-30pt]{caption}
%\usepackage[font=small,skip=0pt]{caption}



%%% Local Variables:
%%% mode: latex
%%% TeX-master: "rr"
%%% End:



%% Symbole de fraction
\newcommand{\Frac}[2]{{\displaystyle \frac{\displaystyle #1}{\displaystyle #2}}}
\newcommand{\Prac}[2]{\displaystyle \genfrac{(}{)}{}{}{\displaystyle #1}{\displaystyle #2}}
\newcommand{\Crac}[2]{\displaystyle \genfrac{[}{]}{}{}{\displaystyle #1}{\displaystyle #2}}

\newcommand{\norme}[1]{\|#1\|}


\newcommand{\HRule}{\rule{\linewidth}{1mm}}

% Fonction mathematiques

\newcommand{\transposee}[1]{{\vphantom{#1}}^{\text{\tiny{\textsf T}}}{#1}}
\newcommand{\argmin}{\mathop{\mathrm{argmin}}}
\newcommand{\argminn}{\mathop{\mathrm{argmin}}}
\newcommand{\lexicomin}{\mathop{\mathrm{lexicomin}}}
%\newcommand{\arg}{\mathop{\mathrm{arg}}}



\DeclareMathOperator{\rot}{rot}
\DeclareMathOperator{\sh}{sh}
\DeclareMathOperator{\ch}{ch}
%\DeclareMathOperator{\th}{th}
\DeclareMathOperator{\arcsh}{arcsh}
\DeclareMathOperator{\argth}{argth}
\DeclareMathOperator{\sign}{sign}


%%The Principal Value Integral symbol
\def\Xint#1{\mathchoice
   {\XXint\displaystyle\textstyle{#1}}%
   {\XXint\textstyle\scriptstyle{#1}}%
   {\XXint\scriptstyle\scriptscriptstyle{#1}}%
   {\XXint\scriptscriptstyle\scriptscriptstyle{#1}}%
   \!\int}
\def\XXint#1#2#3{{\setbox0=\hbox{$#1{#2#3}{\int}$}
     \vcenter{\hbox{$#2#3$}}\kern-.5\wd0}}
\def\ddashint{\Xint=}
\def\dashint{\Xint-}



% macro pour les symbols d'ensemble
%\nbOne
\def\nbOne{{\mathchoice{\rm 1\mskip-4mu l}{\rm 1\mskip-4mu l} {\rm 1 \mskip-4.5mu l}{\rm 1\mskip-5mu l}}}
%
%%  Les ensembles de nombres  C. Fiorio (fiorio@math.tu-berlin.de) 
%
\def\nbR{\ensuremath{\mathrm{I\!R}}} % IR
\def\nbN{\ensuremath{\mathrm{I\!N}}} % IN
\def\nbF{\ensuremath{\mathrm{I\!F}}} % IF
\def\nbH{\ensuremath{\mathrm{I\!H}}} % IH
\def\nbK{\ensuremath{\mathrm{I\!K}}} % IK
\def\nbL{\ensuremath{\mathrm{I\!L}}} % IL
\def\nbM{\ensuremath{\mathrm{I\!M}}} % IM
\def\nbP{\ensuremath{\mathrm{I\!P}}} % IP
%
% \nbOne : 1I : symbol one
\def\nbOne{{\mathchoice {\rm 1\mskip-4mu l} {\rm 1\mskip-4mu l}
{\rm 1\mskip-4.5mu l} {\rm 1\mskip-5mu l}}}
%
% \nbC   :  Nombres Complexes
\def\nbC{{\mathchoice {\setbox0=\hbox{$\displaystyle\rm C$}%
\hbox{\hbox to0pt{\kern0.4\wd0\vrule height0.9\ht0\hss}\box0}}
{\setbox0=\hbox{$\textstyle\rm C$}\hbox{\hbox
to0pt{\kern0.4\wd0\vrule height0.9\ht0\hss}\box0}}
{\setbox0=\hbox{$\scriptstyle\rm C$}\hbox{\hbox
to0pt{\kern0.4\wd0\vrule height0.9\ht0\hss}\box0}}
{\setbox0=\hbox{$\scriptscriptstyle\rm C$}\hbox{\hbox
to0pt{\kern0.4\wd0\vrule height0.9\ht0\hss}\box0}}}}
%
% \nbQ   : Nombres Rationnels Q
\def\nbQ{{\mathchoice {\setbox0=\hbox{$\displaystyle\rm
Q$}\hbox{\raise
0.15\ht0\hbox to0pt{\kern0.4\wd0\vrule height0.8\ht0\hss}\box0}}
{\setbox0=\hbox{$\textstyle\rm Q$}\hbox{\raise
0.15\ht0\hbox to0pt{\kern0.4\wd0\vrule height0.8\ht0\hss}\box0}}
{\setbox0=\hbox{$\scriptstyle\rm Q$}\hbox{\raise
0.15\ht0\hbox to0pt{\kern0.4\wd0\vrule height0.7\ht0\hss}\box0}}
{\setbox0=\hbox{$\scriptscriptstyle\rm Q$}\hbox{\raise
0.15\ht0\hbox to0pt{\kern0.4\wd0\vrule height0.7\ht0\hss}\box0}}}}
%
% \nbT   : T
\def\nbT{{\mathchoice {\setbox0=\hbox{$\displaystyle\rm
T$}\hbox{\hbox to0pt{\kern0.3\wd0\vrule height0.9\ht0\hss}\box0}}
{\setbox0=\hbox{$\textstyle\rm T$}\hbox{\hbox
to0pt{\kern0.3\wd0\vrule height0.9\ht0\hss}\box0}}
{\setbox0=\hbox{$\scriptstyle\rm T$}\hbox{\hbox
to0pt{\kern0.3\wd0\vrule height0.9\ht0\hss}\box0}}
{\setbox0=\hbox{$\scriptscriptstyle\rm T$}\hbox{\hbox
to0pt{\kern0.3\wd0\vrule height0.9\ht0\hss}\box0}}}}
%
% \nbS   : S
\def\nbS{{\mathchoice
{\setbox0=\hbox{$\displaystyle     \rm S$}\hbox{\raise0.5\ht0%
\hbox to0pt{\kern0.35\wd0\vrule height0.45\ht0\hss}\hbox
to0pt{\kern0.55\wd0\vrule height0.5\ht0\hss}\box0}}
{\setbox0=\hbox{$\textstyle        \rm S$}\hbox{\raise0.5\ht0%
\hbox to0pt{\kern0.35\wd0\vrule height0.45\ht0\hss}\hbox
to0pt{\kern0.55\wd0\vrule height0.5\ht0\hss}\box0}}
{\setbox0=\hbox{$\scriptstyle      \rm S$}\hbox{\raise0.5\ht0%
\hboxto0pt{\kern0.35\wd0\vrule height0.45\ht0\hss}\raise0.05\ht0%
\hbox to0pt{\kern0.5\wd0\vrule height0.45\ht0\hss}\box0}}
{\setbox0=\hbox{$\scriptscriptstyle\rm S$}\hbox{\raise0.5\ht0%
\hboxto0pt{\kern0.4\wd0\vrule height0.45\ht0\hss}\raise0.05\ht0%
\hbox to0pt{\kern0.55\wd0\vrule height0.45\ht0\hss}\box0}}}}
%
% \nbZ   : Entiers Relatifs Z
\def\nbZ{{\mathchoice {\hbox{$\sf\textstyle Z\kern-0.4em Z$}}
{\hbox{$\sf\textstyle Z\kern-0.4em Z$}}
{\hbox{$\sf\scriptstyle Z\kern-0.3em Z$}}
{\hbox{$\sf\scriptscriptstyle Z\kern-0.2em Z$}}}}
%%%% fin macro %%%%



\newcommand{\putidx}[1]{\index{#1}\textit{#1}}


%\definecolor{darkgray}{gray}{.25}
\definecolor{gray}{gray}{.5}
\definecolor{lightgray}{gray}{.75}
%\definecolor{gradbegin}{rgb}{0,1,1}
%\definecolor{gradend}{rgb}{0,.1,.95}
%\newcommand{\newtexte}[1]{\textcolor{darkgray} {#1}}
\newcommand{\newtexte}[1]{{#1}}% macro pour les varibales favorites
% normal tangent
\def\n{{\hbox{\tiny{N}}}}
\def\t{{\hbox{\tiny{T}}}}
\def\tone{{\hbox{\tiny{T}}}_{\hbox{\tiny{1}}}}
\def\ttwo{{\hbox{\tiny{T}}}_{\hbox{\tiny{2}}}}
\def\ss{{\hbox{\tiny{S}}}}
\def\nt{\hbox{\tiny{NT}}}
\def\nsf{\hbox{\tiny{\textsf N}}}
\def\tsf{\hbox{\tiny{\textsf T}}}
\def\sigman{\sigma_{\n}}
\def\sigmat{\sigma_{\t}}
\def\sigmant{\sigma_{\nt}}
\def\epsn{\epsilon_{\n}}
\def\epst{\epsilon_{\t}}
\def\epsnt{\epsilon_{\nt}}
\def\eps{\epsilon}
\def\veps{\varepsilon}
\def\sig{\sigma}
\def\Rn{R_{\n}}
\def\Rt{R_{\t}}
\def\cn{c_{\n}}
\def\Cn{C_{\n}}
\def\ct{c_{\t}}
\def\Ct{C_{\t}}
\def\un{u_{\n}}
\def\ut{\buu_{\t}}
\def\uut{u_{\t}}
\def\unc{u_{\n}^c}
\def\utc{\buu_{\t}^c}
\def\vn{v_{\n}}
\def\vt{v_{\t}}
\def\rr{\hbox{\tiny{\textsf R}}}
\def\irr{\hbox{\tiny{\textsf{IR}}}}
\def\rn{r_{\n}}
\def\rt{\brr_{\t}}
\def\rnc{r_{\n}^c}
\def\rtc{\brr_{\t}^c}
\def\trn{\Tilde{r}_{\n}}
\def\trt{\Tilde{\brr}_{\t}}
\def\tr{\Tilde{\brr}}
\def\tv{\Tilde{\bvv}}
\def\vn{v_{\n}}
\def\vt{\bvv_{\t}}
\def\adh{\mathsf{adh}}
\def\adj{\hbox{\tiny{\textsf{adj}}}}
\def\adjc{\hbox{\tiny{\textsf{adjC}}}}
\def\adja{\hbox{\tiny{\textsf{adjA}}}}
\def\cc{\hbox{\tiny{\textsf C}}}
\def\ca{\hbox{\tiny{\textsf A}}}

% domaines et frontieres
\def\om{\Omega}
\def\oma{\Omega^{\alpha}}
\def\omu{\Omega^1\cup \Omega^2}
\def\gc{\Gamma_c}
\def\omt{\omu \cup \gc}
% derivee partielle et gradient et divergence
\def\p{\partial}
\def\grad{\nabla}
\def\div{\mathop{\rm div}\nolimits}
%

%\DeclareTextSymbol{\deg}{T1}{6}
%\def\degre{\mathdegree}
%\newcommand{\degre}{\mathdegree}

\def\etc{\textit{etc}\ldots}
\newcommand{\mdegre}{\hbox{\text{\degre}}}

%\def\nscd{\textsf{\bfseries NSCD}}
%\def\nscd{\textsf{NSCD}}
\newcommand{\nscd}{\textsf{NSCD}}
%\Pisymbol{psy}{212} ou encore \Pisymbol{psy}{228}




%----------------------------------------------------------------------
%             Des chiffres avec des ronds autour
%----------------------------------------------------------------------
\def\nombrecercle#1{\def\taille{0.3}
                \put(0,0){#1}
                \put(0.08,0.08){\circle{\taille}}}



\def\ae#1{\stackrel{\mbox{\scriptsize a.e.}}{#1}}
\def\argmin{\mathop{\rm argmin}}
\def\eqref#1{{\rm (\ref{#1})\/}}
\def\indicfon{\mathord{\rm i}}       %indicator function
\def\p{\mathord{\rm proj}}
\def\N{\mathord{\rm N}}
% \def\prosca#1#2{#1\cdot#2}
\def\prosca#1#2{\langle #1,#2\rangle}
\def\qedtext{\mbox{}\hfill$\Box$}
\def\qedmath{\eqno\Box}

\def\s{{$\mathcal{S}$}}
\def\somme{\mathop{\textstyle\sum}}
\def\somme{\mathop{\textstyle\sum}}
\def\submoins{_{\scriptscriptstyle-}}
\def\subplus{_{\scriptscriptstyle+}}
\def\T{\mathord{\rm T}}

%----------------------------------------------------------------------
%             Macro M Jean 
%----------------------------------------------------------------------

\def\Real{\mbox{I\hspace{-.15em}R}}
\def\Integer{\mbox{I\hspace{-.15em}N}}
\def\Bunit{\mbox{I\hspace{-.15em}B}}
\def\real{\mbox{\scriptsize I\hspace{-.15em}R}}
\def\bunit{\mbox{\scriptsize I\hspace{-.15em}B}}
\def\IL{\mbox{\scriptsize I\hspace{-.15em}L}}
\def\Indic{\mbox{\large $\psi$}}
\def\bfxi{\mbox{$\xi$ \hspace{-1.1em} $\xi$}}
%\def\bfXi{\mbox{$\Xi$ \hspace{-1.1em} $\Xi$}}
\def\RunR{\mathcal R}
\def\RunRN{\mathcal R_{N}}
\def\RunRT{\mathcal R_{T}}
\def\RunS{\mathcal S}
\def\RunSN{\mathcal S_{N}}
\def\RunST{\mathcal S_{T}}
\def\RunU{\mathcal U}
\def\RunUN{\mathcal U_{N}}
\def\RunUT{\mathcal U_{T}}
\def\RunUP{\mathcal U'}
\def\RunUPN{\mathcal U'_{N}}
\def\RunUPT{\mathcal U'_{T}}
\def\RunJ{\mathcal J}
\def\RunW{\mathcal W}
\def\RunF{f}
\def\RunFa{f_{1}}
\def\RunFb{f_{2}}
\def\RunFP{f'}
\def\RunV{v}
\def\RunVP{v'}
\def\EspF{\mathcal F}
\def\EspV{\mathcal V}

\def\N{\mbox{I\hspace{ -.15em}N}}
\def\Z{\mbox{Z\hspace{ -.3em}Z}}
\def\Q{\mbox{l\hspace{ -.47em}Q}}
\def\R{\mbox{l\hspace{ -.15em}R}}
\def\F{\mbox{l\hspace{ -.15em}F}}
\def\E{\mbox{l\hspace{ -.15em}E}}
\def\LMGC90{{\small \it LMGC90 }}
\def\NSCD{{\small \it NSCD }}
\def\CHIC{{\small \it CHIC }}
\def\half{{\frac{_{1}}{^{2}}}}
\def\12T{{\frac{_{1}}{^{2T}}}}

\def\geq{\geqslant}
\def\leq{\leqslant}
\def\ge{\geqslant}
\def\le{\leqslant}


\begingroup
\count0=\time \divide\count0by60 % Hour
\count2=\count0 \multiply\count2by-60 \advance\count2by\time
% Min
\def\2#1{\ifnum#1<10 0\fi\the#1}
\xdef\isodayandtime{\the\year-\2\month-\2\day\space\2{\count0}:%
\2{\count2}}
\endgroup

%---------------------------------------------------------------------
%             Redaction note environnement B. Brogliato
%----------------------------------------------------------------------
\makeatletter

{\newtheorem{ndr1bb}{\textbf{\textsc{Redaction note B.B.}}}[section]}

\newenvironment{ndrbb}%
{%
\noindent\begin{ndr1bb}\hrule\vspace{1em}%
\ttfamily\small
}%
{%
\begin{flushright}%
%\vspace{-1.5em}\ding{111}
\end{flushright}%
\vspace{-1.5em}\hrule
\end{ndr1bb}%
}
%----------------------------------------------------------------------
%             Redaction note environnement V.ACARY
%----------------------------------------------------------------------
% Faut etre fou pour s'amuser a pondre des notes pareilles

{\newtheorem{ndr1va}{\textbf{\textsc{\footnotesize Redaction note V.A.}}}[section]}

\newenvironment{ndrva}%
{%
\noindent\begin{ndr1va}\hrule\vspace{1em}%
\ttfamily\small \  \\
\noindent}%
{%
$ $ \\
\hrule
\end{ndr1va}%
}
\makeatother








% ----------------DEFINITIONS-----------------
% 

 \def\II{\mathop{{\rm I}\mskip-3.0mu{\rm I}}\nolimits}




% -----------------------------------
 \def\c{\mathop{{\rm 1}\mskip-10.0mu{\rm C}}\nolimits}
 \def\C{\mathop{{\rm 1}\mskip-10.0mu{\rm C}}\nolimits}
 \def\ZZ{\mathaccent23Z}
% 

\newcommand{\ie}{{\textit{i.e.}}}


%\def\sgn{\mbox{\rm sgn}}
\DeclareMathOperator{\sgn}{sgn}
\DeclareMathOperator{\proj}{proj}
\DeclareMathOperator{\prox}{prox}
\DeclareMathOperator{\co}{co}


%\newcommand{\RR}{\mbox{\rm $I\!\!R$}}
%\newcommand{\NN}{\mbox{\rm $I\!\!N$}}


\def\RR{\nbR}
\def\NN{\nbN}

% ---------------- MMC -----------------
% 

\newcommand{\contract}{{\,:\,}}

\newcommand{\scontract}{{\,{\Bar\otimes}\,}}
\newcommand{\tcontract}{{\,{\Bar{\Bar{\Bar\otimes}}}\,}}


\newcommand{\DP}[2]{\displaystyle \frac{\partial {#1}}{\partial {#2}}}

\newtheorem{definition}{Definition}
\newtheorem{proposition}{Proposition}
\newtheorem{lemma}{Lemma}

\newtheorem{claim}{Claim}
\newtheorem{remark}{Remark}
\newtheorem{assumption}{Assumption}
\newtheorem{example}{Example}
\newtheorem{conjecture}{Conjecture}
\newtheorem{corollary}{Corollary}
\newtheorem{OP}{OP}
\newtheorem{problem}{Problem}
\newtheorem{theorem}{Theorem}


\def\dt{{\rm d}t}
\def\dv{{\rm d}v}
\def\di{{\rm d}i}
\def\dI{{\rm d}I}
\def\dU{{\rm d}U}


\def\nat{{\hbox{\sf \tiny{nat}}}}
\def\nor{{\hbox{\sf \tiny{nor}}}}
\def\fb{\hbox{\tiny{\textsf FB}}}
\def\vione{{\hbox{\tiny{vi-1}}}}
\def\vitwo{{\hbox{\tiny{vi-2}}}}
\def\qvitwo{{\hbox{\tiny{qvi-2}}}}
\def\mjone{{\hbox{\tiny{mj-1}}}}
\def\mjtwo{{\hbox{\tiny{mj-2}}}}
\def\acone{{\hbox{\tiny{ac-1}}}}
\def\actwo{{\hbox{\tiny{ac-2}}}}





\begingroup
\count0=\time \divide\count0by60 % Hour
\count2=\count0 \multiply\count2by-60 \advance\count2by\time
% Min
\def\2#1{\ifnum#1<10 0\fi\the#1}
\xdef\isodayandtime{\the\year-\2\month-\2\day\space\2{\count0}:%
\2{\count2}}
\endgroup
%%%% fin macro %%%%




%%
\begin{document}
%%
\RRNo{123456789}
%\makeRR   % cas d'un rapport de recherche
\makeRT
%% a partir d'ici, chacun fait comme il le souhaite

\newpage
\tableofcontents
\newpage

\section{Introduction}
The computation of numerical solutions of miscellaneous nonsmooth physical
problems may need the expression and the computation of Clarke Generalized
gradients \cite{Clarke1975}.  Code writing for such jacobians can be tedious
and error prone. Errors in the computation may not be obvious to detect when
the jacobian is used in nonsmooth Newton method.


\section{The Method}


The method:

1. Express nonsmooth function in a computer algebra system (Maple, Sagemath, Sympy)

2. Compute the jacobian on differentiable points.

3. Express generalized jacobian as a piecewise function : continuous : ``normal jacobian'',  non differentiable point : a limit

4. perform common subexpression elimination up to functions sqrt, Abs, Max, Heaviside, etc.

5. produce c code with static assertions (value analysis fails without them)

6. static verification with frama-c:

output a proof of: limited domain (physically admissible domain) for inputs ==> is\_finite(result)

\section{Examples}

\subsection{Euclidian norm function}

\subsubsection{code generation from the mathematical definition}
\begin{equation}
  \begin{array}{ll}
  \left [\begin{array}{c}
    x\\
    y
    \end{array} \right] \mapsto \sqrt{x^2+y^2}
  \end{array}
\end{equation}

\begin{equation}
  \begin{array}{ll}
    \left [
      \begin{array}{c}
        x\\
        y
      \end{array} \right]
    \mapsto \left\{
        \begin{array}{ll}
          \left[
            \begin{array}{c}
            \frac{x}{\sqrt{x^2+y^2}}\\
            \frac{y}{\sqrt{x^2+y^2}}
          \end{array} \right] & \mbox{if } (x, y) \in \RR^{\star 2} \\
        \{ v \in \RR^2, ||v|| = 1 \} & \mbox{if } (x, y) = (0, 0)
        
      \end{array}
    \right.
  \end{array}
\end{equation}

\begin{listing}[H] 
  \begin{minted}[frame=single]{python}
    from sympy import Symbol, Matrix, sqrt
    x = Symbol('x', real=True)
    y = Symbol('y', real=True)
    norm = Matrix([sqrt(x*x + y*y)])
  \end{minted}
  \caption{The two-norm function over $\RR^2$ with Sympy}
\end{listing}


The smooth jacobian

\begin{listing}[H] 
  \begin{minted}[frame=single]{python}
    v = Matrix([x, y])
    J_ = f.jacobian(v)
  \end{minted}
  \caption{the smooth jacobian}
\end{listing}

In zero
\begin{listing}[H] 
  \begin{minted}[frame=single]{python}
    from sympy import limit
    def lim0(expr):
        return limit(expr.subs(x,t).subs(y,t), t, 0)
  \end{minted}
  \caption{a limit}
\end{listing}

\begin{listing}[H] 
  \begin{minted}[frame=single]{python}
    from sympy import Piecewise
    J = Matrix(J_.shape[0], J_.shape[1], 
           lambda i, j: Piecewise(
               (J_[i, j], v.norm() > 0.),
               (lim0(J_[i, j]), v.norm() <= 0.)))
  \end{minted}
  \caption{a limit}
\end{listing}


\begin{equation}
  \begin{array}{ll}
    \left [
      \begin{array}{c}
        x\\
        y
      \end{array} \right]
    \mapsto \left\{
        \begin{array}{ll}
          \left[
            \begin{array}{c}
            \frac{x}{\sqrt{x^2+y^2}}\\
            \frac{y}{\sqrt{x^2+y^2}}
          \end{array} \right] & \mbox{if } (x, y) \in \RR^{\star 2} \\
        \left[
          \begin{array}{c}
            \frac{\sqrt{2}}{2}\\
            \frac{\sqrt{2}}{2}
          \end{array} \right] & \mbox{if } (x, y) = (0, 0)
        
      \end{array}
    \right.
  \end{array}
\end{equation}


\begin{listing}[H] 
  \begin{minted}[frame=single]{c}
void norm2d_jacobian(
    double x,
    double y,
    double *result)
{
    double x1 = 0.;
    int x4 = 0;
    double x2 = 0.;
    double x3 = 0.;
    x1 = sqrt(x*x + y*y);
    x4 = x1 > 0.;
    int x5 = 0;
    x5 = x1 <= 0.;
    if (x4)
    {
        x2 = 1.0/x1;
        x3 = 1.0*x2;
    }
    if (x4)
    {
        result[0] = x*x3;
    }
    else if (x5)
    {
        result[0] = 0.707106781186547524400844362104849039284835937688474;
    }
    if (x4)
    {
        result[1] = x3*y;
    }
    else if (x5)
    {
        result[1] = 0.707106781186547524400844362104849039284835937688474;
    }
}
\end{minted}
\caption{a limit}
\end{listing}

\subsubsection{this function may overflow}

\subsubsection{a corrected jacobian}

\section{Jacobians}

The Heaviside function is noted $\theta$

The $\epsilon$ parameter is important to avoid infinites or not a number values. 
The verification is performed by the value analysis.

\subsection{The Alart Curnier and the Jean Moreau functions}

Given the following notations:

\begin{equation}
  \begin{array}{c}
    \Delta_n = r_n -\rho_n  u_n\\
    \Delta_{t1} = t_{t1} - \rho_{t1}  u_{t1}\\
    \Delta_{t2} = t_{t2} - \rho_{t2}  u_{t2}\\
    \Delta_{t} = \left [
      \begin{array}{c}
        \Delta_{t1}\\
        \Delta_{t2}
      \end{array} \right]
  \end{array}
\end{equation}

\subsubsection{Alart Curnier}

The Alart Curnier function $F_{AC}$ is written as:

\begin{equation}
  F_{AC} = \left[\begin{matrix}r_{n} - \max\left(\Delta_{n}, \epsilon\right)\\\begin{cases} \rho_{{t1}} u_{{t1}} & \text{for}\: \lVert {\Delta_{t} \rVert} \leq \mu \max\left(\Delta_{n}, \epsilon\right) \\- \frac{\Delta_{{t1}} \mu \max\left(\Delta_{n}, \epsilon\right)}{\lVert {\Delta_{t} \rVert}} + r_{{t1}} & \text{for}\: \lVert {\Delta_{t} \rVert} > \mu \max\left(\Delta_{n}, \epsilon\right) \end{cases}\\\begin{cases} \rho_{{t2}} u_{{t2}} & \text{for}\: \lVert {\Delta_{t} \rVert} \leq \mu \max\left(\Delta_{n}, \epsilon\right) \\- \frac{\Delta_{{t2}} \mu \max\left(\Delta_{n}, \epsilon\right)}{\lVert {\Delta_{t} \rVert}} + r_{{t2}} & \text{for}\: \lVert {\Delta_{t} \rVert} > \mu \max\left(\Delta_{n}, \epsilon\right) \end{cases}\end{matrix}\right]
\end{equation}

We have 
\begin{equation}
  A = \left[ A_0 A_1 A_2 \right]
\end{equation}

with:
\begin{equation}
  A_0 = \left[\begin{matrix}\rho_{n} \theta\left(- \epsilon - \rho_{n} u_{n} + r_{n}\right)\\\begin{cases} 0 & \text{for}\: \lVert {\Delta_{t} \rVert} \leq \mu \max\left(\Delta_{n}, \epsilon\right) \\\frac{\Delta_{{t1}} \rho_{n} \mu}{\lVert {\Delta_{t} \rVert}} \theta\left(- \epsilon - \rho_{n} u_{n} + r_{n}\right) & \text{for}\: \lVert {\Delta_{t} \rVert} > \mu \max\left(\Delta_{n}, \epsilon\right) \end{cases}\\\begin{cases} 0 & \text{for}\: \lVert {\Delta_{t} \rVert} \leq \mu \max\left(\Delta_{n}, \epsilon\right) \\\frac{\Delta_{{t2}} \rho_{n} \mu}{\lVert {\Delta_{t} \rVert}} \theta\left(- \epsilon - \rho_{n} u_{n} + r_{n}\right) & \text{for}\: \lVert {\Delta_{t} \rVert} > \mu \max\left(\Delta_{n}, \epsilon\right) \end{cases}\end{matrix}\right]
\end{equation}

\begin{equation}
  A_1 = \left[\begin{matrix}0\\\begin{cases} \rho_{{t1}} & \text{for}\: \lVert {\Delta_{t} \rVert} \leq \mu \max\left(\Delta_{n}, \epsilon\right) \\\frac{\rho_{{t1}} \mu \max\left(\Delta_{n}, \epsilon\right)}{\lVert {\Delta_{t} \rVert}^{3}} \left(- \Delta_{{t1}}^{2} + \lVert {\Delta_{t} \rVert}^{2}\right) & \text{for}\: \lVert {\Delta_{t} \rVert} > \mu \max\left(\Delta_{n}, \epsilon\right) \end{cases}\\\begin{cases} 0 & \text{for}\: \lVert {\Delta_{t} \rVert} \leq \mu \max\left(\Delta_{n}, \epsilon\right) \\- \frac{\Delta_{{t1}} \Delta_{{t2}} \rho_{{t1}} \mu \max\left(\Delta_{n}, \epsilon\right)}{\lVert {\Delta_{t} \rVert}^{3}} & \text{for}\: \lVert {\Delta_{t} \rVert} > \mu \max\left(\Delta_{n}, \epsilon\right) \end{cases}\end{matrix}\right]
\end{equation}

\begin{equation}
  A_2 = \left[\begin{matrix}0\\\begin{cases} 0 & \text{for}\: \lVert {\Delta_{t} \rVert} \leq \mu \max\left(\Delta_{n}, \epsilon\right) \\- \frac{\Delta_{{t1}} \Delta_{{t2}} \rho_{{t2}} \mu \max\left(\Delta_{n}, \epsilon\right)}{\lVert {\Delta_{t} \rVert}^{3}} & \text{for}\: \lVert {\Delta_{t} \rVert} > \mu \max\left(\Delta_{n}, \epsilon\right) \end{cases}\\\begin{cases} \rho_{{t2}} & \text{for}\: \lVert {\Delta_{t} \rVert} \leq \mu \max\left(\Delta_{n}, \epsilon\right) \\\frac{\rho_{{t2}} \mu \max\left(\Delta_{n}, \epsilon\right)}{\lVert {\Delta_{t} \rVert}^{3}} \left(- \Delta_{{t2}}^{2} + \lVert {\Delta_{t} \rVert}^{2}\right) & \text{for}\: \lVert {\Delta_{t} \rVert} > \mu \max\left(\Delta_{n}, \epsilon\right) \end{cases}\end{matrix}\right]
\end{equation}

and

\begin{equation}
  B = \left[ B_0 B_1 B_2 \right]
\end{equation}

with:
\begin{equation}
  B_0 = \input{fac_B0.tex}
\end{equation}

\begin{equation}
  B_1 = \input{fac_B1.tex}
\end{equation}

\begin{equation}
  B_2 = \left[\begin{matrix}0\\\begin{cases} 0 & \text{for}\: \lVert {\Delta_{t} \rVert} \leq \mu \max\left(\Delta_{n}, \epsilon\right) \\\frac{\Delta_{{t1}} \Delta_{{t2}} \mu \max\left(\Delta_{n}, \epsilon\right)}{\lVert {\Delta_{t} \rVert}^{3}} & \text{for}\: \lVert {\Delta_{t} \rVert} > \mu \max\left(\Delta_{n}, \epsilon\right) \end{cases}\\\begin{cases} 0 & \text{for}\: \lVert {\Delta_{t} \rVert} \leq \mu \max\left(\Delta_{n}, \epsilon\right) \\\frac{\Delta_{{t2}}^{2} \mu \max\left(\Delta_{n}, \epsilon\right)}{\lVert {\Delta_{t} \rVert}^{3}} + 1 - \frac{\mu \max\left(\Delta_{n}, \epsilon\right)}{\lVert {\Delta_{t} \rVert}} & \text{for}\: \lVert {\Delta_{t} \rVert} > \mu \max\left(\Delta_{n}, \epsilon\right) \end{cases}\end{matrix}\right]
\end{equation}

\subsubsection{Jean Moreau}
The Jean Moreau function $F_{JM}$ is written as:

\begin{equation}
  F_{JM} = \input{fJM.tex}
\end{equation}

We have 
\begin{equation}
  A = \left[ A_0 A_1 A_2 \right]
\end{equation}

with:
\begin{equation}
  A_0 = \left[\begin{matrix}\rho_{n} \theta\left(- \epsilon - \rho_{n} u_{n} + r_{n}\right)\\0\\0\end{matrix}\right]
\end{equation}

\begin{equation}
  A_1 = \left[\begin{matrix}0\\\begin{cases} \rho_{{t1}} & \text{for}\: \lVert {\Delta_{t} \rVert} \leq \mu \max\left(\epsilon, r_{n}\right) \\\frac{\rho_{{t1}} \mu \max\left(\epsilon, r_{n}\right)}{\lVert {\Delta_{t} \rVert}^{3}} \left(- \Delta_{{t1}}^{2} + \lVert {\Delta_{t} \rVert}^{2}\right) & \text{for}\: \lVert {\Delta_{t} \rVert} > \mu \max\left(\epsilon, r_{n}\right) \end{cases}\\\begin{cases} 0 & \text{for}\: \lVert {\Delta_{t} \rVert} \leq \mu \max\left(\epsilon, r_{n}\right) \\- \frac{\Delta_{{t1}} \Delta_{{t2}} \rho_{{t1}} \mu \max\left(\epsilon, r_{n}\right)}{\lVert {\Delta_{t} \rVert}^{3}} & \text{for}\: \lVert {\Delta_{t} \rVert} > \mu \max\left(\epsilon, r_{n}\right) \end{cases}\end{matrix}\right]
\end{equation}

\begin{equation}
  A_2 = \left[\begin{matrix}0\\\begin{cases} 0 & \text{for}\: \lVert {\Delta_{t} \rVert} \leq \mu \max\left(\epsilon, r_{n}\right) \\- \frac{\Delta_{{t1}} \Delta_{{t2}} \rho_{{t2}} \mu \max\left(\epsilon, r_{n}\right)}{\lVert {\Delta_{t} \rVert}^{3}} & \text{for}\: \lVert {\Delta_{t} \rVert} > \mu \max\left(\epsilon, r_{n}\right) \end{cases}\\\begin{cases} \rho_{{t2}} & \text{for}\: \lVert {\Delta_{t} \rVert} \leq \mu \max\left(\epsilon, r_{n}\right) \\\frac{\rho_{{t2}} \mu \max\left(\epsilon, r_{n}\right)}{\lVert {\Delta_{t} \rVert}^{3}} \left(- \Delta_{{t2}}^{2} + \lVert {\Delta_{t} \rVert}^{2}\right) & \text{for}\: \lVert {\Delta_{t} \rVert} > \mu \max\left(\epsilon, r_{n}\right) \end{cases}\end{matrix}\right]
\end{equation}

and

\begin{equation}
  B = \left[ B_0 B_1 B_2 \right]
\end{equation}

with:
\begin{equation}
  B_0 = \left[\begin{matrix}- \theta\left(- \epsilon - \rho_{n} u_{n} + r_{n}\right) + 1\\\begin{cases} 0 & \text{for}\: \lVert {\Delta_{t} \rVert} \leq \mu \max\left(\epsilon, r_{n}\right) \\- \frac{\Delta_{{t1}} \mu \theta\left(- \epsilon + r_{n}\right)}{\lVert {\Delta_{t} \rVert}} & \text{for}\: \lVert {\Delta_{t} \rVert} > \mu \max\left(\epsilon, r_{n}\right) \end{cases}\\\begin{cases} 0 & \text{for}\: \lVert {\Delta_{t} \rVert} \leq \mu \max\left(\epsilon, r_{n}\right) \\- \frac{\Delta_{{t2}} \mu \theta\left(- \epsilon + r_{n}\right)}{\lVert {\Delta_{t} \rVert}} & \text{for}\: \lVert {\Delta_{t} \rVert} > \mu \max\left(\epsilon, r_{n}\right) \end{cases}\end{matrix}\right]
\end{equation}

\begin{equation}
  B_1 = \input{fJM_B1.tex}
\end{equation}

\begin{equation}
  B_2 = \left[\begin{matrix}0\\\begin{cases} 0 & \text{for}\: \lVert {\Delta_{t} \rVert} \leq \mu \max\left(\epsilon, r_{n}\right) \\\frac{\Delta_{{t1}} \Delta_{{t2}} \mu \max\left(\epsilon, r_{n}\right)}{\lVert {\Delta_{t} \rVert}^{3}} & \text{for}\: \lVert {\Delta_{t} \rVert} > \mu \max\left(\epsilon, r_{n}\right) \end{cases}\\\begin{cases} 0 & \text{for}\: \lVert {\Delta_{t} \rVert} \leq \mu \max\left(\epsilon, r_{n}\right) \\\frac{\Delta_{{t2}}^{2} \mu \max\left(\epsilon, r_{n}\right)}{\lVert {\Delta_{t} \rVert}^{3}} + 1 - \frac{\mu \max\left(\epsilon, r_{n}\right)}{\lVert {\Delta_{t} \rVert}} & \text{for}\: \lVert {\Delta_{t} \rVert} > \mu \max\left(\epsilon, r_{n}\right) \end{cases}\end{matrix}\right]
\end{equation}

\subsection{The normal map function}

With the notations:
\begin{equation}
  \begin{array}{c}
    \Delta_{t1} = \mu u_{t1}-r_{t1}\\
    \Delta_{t2} = \mu u_{t2}-r_{t2}\\
    \Delta_{t} = \left [
      \begin{array}{c}
        \Delta_{t1}\\
        \Delta_{t2}
      \end{array} \right]
  \end{array}
\end{equation}

\begin{equation}
  F_{nat} = \left[\begin{matrix}\mu r_{n} - \frac{1}{2} \left(\max\left(0, \lambda_{1}\right) + \max\left(0, \lambda_{2}\right)\right)\\\begin{cases} - \frac{\Delta_{1} \left(\max\left(0, \lambda_{1}\right) - \max\left(0, \lambda_{2}\right)\right)}{2 \lVert {\Delta} \rVert} + r_{{t1}} & \text{for}\: \lVert {\Delta} \rVert > \epsilon \\r_{{t1}} & \text{for}\: \lVert {\Delta} \rVert \leq \epsilon \end{cases}\\\begin{cases} - \frac{\Delta_{2} \left(\max\left(0, \lambda_{1}\right) - \max\left(0, \lambda_{2}\right)\right)}{2 \lVert {\Delta} \rVert} + r_{{t2}} & \text{for}\: \lVert {\Delta} \rVert > \epsilon \\r_{{t2}} + \frac{1}{2} \left(\max\left(0, \lambda_{1}\right) - \max\left(0, \lambda_{2}\right)\right) & \text{for}\: \lVert {\Delta} \rVert \leq \epsilon \end{cases}\end{matrix}\right]
\end{equation}

\begin{equation}
  A_0 = \left[\begin{matrix}\frac{1}{2} \left(\theta\left(\lambda_{1}\right) + \theta\left(\lambda_{2}\right)\right)\\\begin{cases} \frac{\Delta_{1} \left(\theta\left(\lambda_{1}\right) - \theta\left(\lambda_{2}\right)\right)}{2 \lVert {\Delta} \rVert} & \text{for}\: \lVert {\Delta} \rVert > \epsilon \\0 & \text{for}\: \lVert {\Delta} \rVert \leq \epsilon \end{cases}\\\begin{cases} \frac{\Delta_{2} \left(\theta\left(\lambda_{1}\right) - \theta\left(\lambda_{2}\right)\right)}{2 \lVert {\Delta} \rVert} & \text{for}\: \lVert {\Delta} \rVert > \epsilon \\- \frac{1}{2} \left(\theta\left(\lambda_{1}\right) - \theta\left(\lambda_{2}\right)\right) & \text{for}\: \lVert {\Delta} \rVert \leq \epsilon \end{cases}\end{matrix}\right]
\end{equation}

\begin{equation}
  A_{01} = \input{A01_fnat.tex}
\end{equation}

\begin{equation}
  A_{02} = \begin{cases} \frac{\sqrt{2} \mu}{2} \theta\left(\lambda^{{\prime\prime}}_{1}\right) & \text{for}\: \lVert {r_{{t}} \rVert} \leq \epsilon \wedge \lVert {u_{{t}} \rVert} \leq \epsilon \\\frac{\mu r_{{t2}} \theta\left(\lambda^{{\prime}}_{1}\right)}{\lVert {r_{{t}} \rVert}} & \text{for}\: \lVert {\Delta} \rVert \leq \epsilon \\- \frac{\theta\left(\lambda_{1}\right)}{2} \left(- \frac{\Delta_{2} \mu}{\lVert {\Delta} \rVert} - \frac{\mu u_{{t2}}}{\lVert {u_{{t}} \rVert}}\right) - \frac{\theta\left(\lambda_{2}\right)}{2} \left(\frac{\Delta_{2} \mu}{\lVert {\Delta} \rVert} - \frac{\mu u_{{t2}}}{\lVert {u_{{t}} \rVert}}\right) & \text{for}\: \lVert {\Delta} \rVert > \epsilon \wedge \lVert {u_{{t}} \rVert} > \epsilon \\\frac{\mu}{4 \lVert {r_{{t}} \rVert}} \left(- 2 r_{{t2}} \theta\left(\lambda^{{\prime}}_{1}\right) + 2 r_{{t2}} \theta\left(\lambda^{{\prime}}_{2}\right) + \sqrt{2 r_{{t1}}^{2} + 2 r_{{t2}}^{2}} \theta\left(\lambda^{{\prime}}_{1}\right) + \sqrt{2 r_{{t1}}^{2} + 2 r_{{t2}}^{2}} \theta\left(\lambda^{{\prime}}_{2}\right)\right) & \text{for}\: \lVert {u_{{t}} \rVert} \leq \epsilon \wedge \lVert {\Delta} \rVert > \epsilon \wedge \lVert {r_{{t}} \rVert} > \epsilon \end{cases}
\end{equation}

\begin{equation}
  A_{11} = \begin{cases} \frac{1.0 \mu \left(\mu - 1\right)^{2} \theta\left(\lambda^{{\prime\prime}}_{1}\right)}{2 \mu^{2} - 2 \mu + 1} & \text{for}\: \lVert {r_{{t}} \rVert} \leq \epsilon \wedge \lVert {u_{{t}} \rVert} \leq \epsilon \\\frac{\mu r_{{t1}}^{2} \theta\left(\lambda^{{\prime}}_{1}\right)}{\lVert {r_{{t}} \rVert}^{2}} & \text{for}\: \lVert {\Delta} \rVert \leq \epsilon \wedge \lVert {r_{{t}} \rVert} > \epsilon \wedge \lVert {u_{{t}} \rVert} > \epsilon \\- \frac{\Delta_{1} \theta\left(\lambda_{1}\right)}{2 \lVert {\Delta} \rVert} \left(- \frac{\Delta_{1} \mu}{\lVert {\Delta} \rVert} - \frac{\mu u_{{t1}}}{\lVert {u_{{t}} \rVert}}\right) + \frac{\Delta_{1} \theta\left(\lambda_{2}\right)}{2 \lVert {\Delta} \rVert} \left(\frac{\Delta_{1} \mu}{\lVert {\Delta} \rVert} - \frac{\mu u_{{t1}}}{\lVert {u_{{t}} \rVert}}\right) - \frac{\max\left(0, \lambda_{1}\right)}{2} \left(- \frac{\Delta_{1}^{2} \mu}{\lVert {\Delta} \rVert^{3}} + \frac{\mu}{\lVert {\Delta} \rVert}\right) - \frac{\max\left(0, \lambda_{2}\right)}{2} \left(\frac{\Delta_{1}^{2} \mu}{\lVert {\Delta} \rVert^{3}} - \frac{\mu}{\lVert {\Delta} \rVert}\right) & \text{for}\: \lVert {\Delta} \rVert > \epsilon \wedge \lVert {u_{{t}} \rVert} > \epsilon \\\frac{0.25 \mu}{\lVert {r_{{t}} \rVert}^{3}} \left(2.0 \lVert {r_{{t}} \rVert} r_{{t1}}^{2} \theta\left(\lambda^{{\prime}}_{1}\right) + 2.0 \lVert {r_{{t}} \rVert} r_{{t1}}^{2} \theta\left(\lambda^{{\prime}}_{2}\right) - \sqrt{2} r_{{t1}}^{3} \theta\left(\lambda^{{\prime}}_{1}\right) + \sqrt{2} r_{{t1}}^{3} \theta\left(\lambda^{{\prime}}_{2}\right) - \sqrt{2} r_{{t1}} r_{{t2}}^{2} \theta\left(\lambda^{{\prime}}_{1}\right) + \sqrt{2} r_{{t1}} r_{{t2}}^{2} \theta\left(\lambda^{{\prime}}_{2}\right) - 2.0 r_{{t2}}^{2} \max\left(0, \lambda^{{\prime}}_{1}\right) + 2.0 r_{{t2}}^{2} \max\left(0, \lambda^{{\prime}}_{2}\right)\right) & \text{for}\: \lVert {u_{{t}} \rVert} \leq \epsilon \wedge \lVert {\Delta} \rVert > \epsilon \wedge \lVert {r_{{t}} \rVert} > \epsilon \end{cases}
\end{equation}

\begin{equation}
  A_{12} = \begin{cases} \frac{\mu^{2} \left(\mu - 1\right) \theta\left(\lambda^{{\prime\prime}}_{1}\right)}{2 \mu^{2} - 2 \mu + 1} & \text{for}\: \lVert {r_{{t}} \rVert} \leq \epsilon \wedge \lVert {u_{{t}} \rVert} \leq \epsilon \\\frac{\mu r_{{t1}} r_{{t2}} \theta\left(\lambda^{{\prime}}_{1}\right)}{\lVert {r_{{t}} \rVert}^{2}} & \text{for}\: \lVert {\Delta} \rVert \leq \epsilon \\\frac{\Delta_{1} \Delta_{2} \mu \max\left(0, \lambda_{1}\right)}{2 \lVert {\Delta} \rVert^{3}} - \frac{\Delta_{1} \Delta_{2} \mu \max\left(0, \lambda_{2}\right)}{2 \lVert {\Delta} \rVert^{3}} - \frac{\Delta_{1} \theta\left(\lambda_{1}\right)}{2 \lVert {\Delta} \rVert} \left(- \frac{\Delta_{2} \mu}{\lVert {\Delta} \rVert} - \frac{\mu u_{{t2}}}{\lVert {u_{{t}} \rVert}}\right) + \frac{\Delta_{1} \theta\left(\lambda_{2}\right)}{2 \lVert {\Delta} \rVert} \left(\frac{\Delta_{2} \mu}{\lVert {\Delta} \rVert} - \frac{\mu u_{{t2}}}{\lVert {u_{{t}} \rVert}}\right) & \text{for}\: \lVert {\Delta} \rVert > \epsilon \wedge \lVert {u_{{t}} \rVert} > \epsilon \\\frac{\mu r_{{t1}}}{4 \lVert {r_{{t}} \rVert}^{3}} \left(2 \lVert {r_{{t}} \rVert} r_{{t2}} \theta\left(\lambda^{{\prime}}_{1}\right) + 2 \lVert {r_{{t}} \rVert} r_{{t2}} \theta\left(\lambda^{{\prime}}_{2}\right) - \sqrt{2} r_{{t1}}^{2} \theta\left(\lambda^{{\prime}}_{1}\right) + \sqrt{2} r_{{t1}}^{2} \theta\left(\lambda^{{\prime}}_{2}\right) - \sqrt{2} r_{{t2}}^{2} \theta\left(\lambda^{{\prime}}_{1}\right) + \sqrt{2} r_{{t2}}^{2} \theta\left(\lambda^{{\prime}}_{2}\right) + 2 r_{{t2}} \max\left(0, \lambda^{{\prime}}_{1}\right) - 2 r_{{t2}} \max\left(0, \lambda^{{\prime}}_{2}\right)\right) & \text{for}\: \lVert {u_{{t}} \rVert} \leq \epsilon \wedge \lVert {\Delta} \rVert > \epsilon \wedge \lVert {r_{{t}} \rVert} > \epsilon \end{cases}
\end{equation}

\begin{equation}
  A_{21} = \input{A21_fnat.tex}
\end{equation}

\begin{equation}
  A_{22} = \input{A22_fnat.tex}
\end{equation}

\begin{equation}
  B_0 = \input{B0_fnat.tex}
\end{equation}

\begin{equation}
  B_{01} = \begin{cases} 0.0 & \text{for}\: \left(\lVert {\Delta} \rVert \leq \epsilon \vee \lVert {r_{{t}} \rVert} \leq \epsilon\right) \wedge \left(\lVert {\Delta} \rVert \leq \epsilon \vee \lVert {u_{{t}} \rVert} \leq \epsilon\right) \wedge \left(\lVert {r_{{t}} \rVert} \leq \epsilon \vee \lVert {r_{{t}} \rVert} > \epsilon\right) \wedge \left(\lVert {r_{{t}} \rVert} \leq \epsilon \vee \lVert {u_{{t}} \rVert} > \epsilon\right) \wedge \left(\lVert {u_{{t}} \rVert} \leq \epsilon \vee \lVert {r_{{t}} \rVert} > \epsilon\right) \wedge \left(\lVert {u_{{t}} \rVert} \leq \epsilon \vee \lVert {u_{{t}} \rVert} > \epsilon\right) \\- \frac{\Delta_{1} \theta\left(\lambda_{1}\right)}{2 \lVert {\Delta} \rVert} + \frac{\Delta_{1} \theta\left(\lambda_{2}\right)}{2 \lVert {\Delta} \rVert} & \text{for}\: \lVert {\Delta} \rVert > \epsilon \wedge \lVert {r_{{t}} \rVert} > \epsilon \\- \frac{u_{{t1}}}{2 \lVert {u_{{t}} \rVert}} \left(- 1.0 \theta\left(\lambda^{{\prime\prime}}_{1}\right) + \theta\left(- 2.0 \lVert {u_{{t}} \rVert} \mu + \mu r_{n} - 1.0 u_{n}\right)\right) & \text{for}\: \lVert {r_{{t}} \rVert} \leq \epsilon \wedge \lVert {\Delta} \rVert > \epsilon \wedge \lVert {u_{{t}} \rVert} > \epsilon \end{cases}
\end{equation}

\begin{equation}
  B_{02} = \input{B02_fnat.tex}
\end{equation}

\begin{equation}
  B_{11} = \begin{cases} - \frac{\Delta_{1}^{2} \theta\left(\lambda_{1}\right)}{2 \lVert {\Delta} \rVert^{2}} - \frac{\Delta_{1}^{2} \theta\left(\lambda_{2}\right)}{2 \lVert {\Delta} \rVert^{2}} - \frac{\max\left(0, \lambda_{2}\right)}{2} \left(- \frac{\Delta_{1}^{2}}{\lVert {\Delta} \rVert^{3}} + \frac{1}{\lVert {\Delta} \rVert}\right) - \frac{\max\left(0, \lambda_{1}\right)}{2} \left(\frac{\Delta_{1}^{2}}{\lVert {\Delta} \rVert^{3}} - \frac{1}{\lVert {\Delta} \rVert}\right) + 1 & \text{for}\: \lVert {\Delta} \rVert > \epsilon \\1 & \text{for}\: \lVert {\Delta} \rVert \leq \epsilon \end{cases}
\end{equation}

\begin{equation}
  B_{12} = \begin{cases} - \frac{\Delta_{1} \Delta_{2} \theta\left(\lambda_{1}\right)}{2 \lVert {\Delta} \rVert^{2}} - \frac{\Delta_{1} \Delta_{2} \theta\left(\lambda_{2}\right)}{2 \lVert {\Delta} \rVert^{2}} - \frac{\Delta_{1} \Delta_{2} \max\left(0, \lambda_{1}\right)}{2 \lVert {\Delta} \rVert^{3}} + \frac{\Delta_{1} \Delta_{2} \max\left(0, \lambda_{2}\right)}{2 \lVert {\Delta} \rVert^{3}} & \text{for}\: \lVert {\Delta} \rVert > \epsilon \\0 & \text{for}\: \lVert {\Delta} \rVert \leq \epsilon \end{cases}
\end{equation}

\begin{equation}
  B_{21} = \begin{cases} - \frac{1.0 \mu \left(\mu - 1.0\right) \theta\left(\lambda^{{\prime\prime}}_{1}\right)}{2 \mu^{2} - 2 \mu + 1} & \text{for}\: \lVert {r_{{t}} \rVert} \leq \epsilon \wedge \lVert {u_{{t}} \rVert} \leq \epsilon \\- \frac{1.0 r_{{t1}} r_{{t2}} \theta\left(\lambda^{{\prime}}_{1}\right)}{\lVert {r_{{t}} \rVert}^{2}} & \text{for}\: \lVert {\Delta} \rVert \leq \epsilon \wedge \lVert {r_{{t}} \rVert} > \epsilon \wedge \lVert {u_{{t}} \rVert} > \epsilon \\- \frac{\Delta_{1} \Delta_{2} \theta\left(\lambda_{1}\right)}{2 \lVert {\Delta} \rVert^{2}} - \frac{\Delta_{1} \Delta_{2} \theta\left(\lambda_{2}\right)}{2 \lVert {\Delta} \rVert^{2}} - \frac{\Delta_{1} \Delta_{2} \max\left(0, \lambda_{1}\right)}{2 \lVert {\Delta} \rVert^{3}} + \frac{\Delta_{1} \Delta_{2} \max\left(0, \lambda_{2}\right)}{2 \lVert {\Delta} \rVert^{3}} & \text{for}\: \lVert {\Delta} \rVert > \epsilon \wedge \lVert {r_{{t}} \rVert} > \epsilon \\- \frac{1.0 u_{{t1}} u_{{t2}}}{\lVert {u_{{t}} \rVert}^{3} \mu} \left(\frac{\mu \theta\left(\lambda^{{\prime\prime}}_{1}\right)}{2} \lVert {u_{{t}} \rVert} + \frac{\mu}{2} \lVert {u_{{t}} \rVert} \theta\left(- 2 \lVert {u_{{t}} \rVert} \mu + \lambda^{{\prime\prime}}_{1}\right) - \frac{\max\left(0, \lambda^{{\prime\prime}}_{1}\right)}{2} + \frac{1}{2} \max\left(0, - 2 \lVert {u_{{t}} \rVert} \mu + \lambda^{{\prime\prime}}_{1}\right)\right) & \text{for}\: \lVert {r_{{t}} \rVert} \leq \epsilon \wedge \lVert {\Delta} \rVert > \epsilon \wedge \lVert {u_{{t}} \rVert} > \epsilon \end{cases}
\end{equation}

\begin{equation}
  B_{22} = \begin{cases} \frac{\mu \theta\left(\lambda^{{\prime\prime}}_{1}\right)}{\sqrt{2 \mu^{2} - 2 \mu + 1}} + 1 & \text{for}\: \lVert {r_{{t}} \rVert} \leq \epsilon \wedge \lVert {u_{{t}} \rVert} \leq \epsilon \\1 + \frac{r_{{t2}} \theta\left(\lambda^{{\prime}}_{1}\right)}{\lVert {r_{{t}} \rVert}} & \text{for}\: \lVert {\Delta} \rVert \leq \epsilon \wedge \lVert {r_{{t}} \rVert} > \epsilon \wedge \lVert {u_{{t}} \rVert} > \epsilon \\- \frac{\Delta_{2}^{2} \theta\left(\lambda_{1}\right)}{2 \lVert {\Delta} \rVert^{2}} - \frac{\Delta_{2}^{2} \theta\left(\lambda_{2}\right)}{2 \lVert {\Delta} \rVert^{2}} - \frac{\max\left(0, \lambda_{2}\right)}{2} \left(- \frac{\Delta_{2}^{2}}{\lVert {\Delta} \rVert^{3}} + \frac{1}{\lVert {\Delta} \rVert}\right) - \frac{\max\left(0, \lambda_{1}\right)}{2} \left(\frac{\Delta_{2}^{2}}{\lVert {\Delta} \rVert^{3}} - \frac{1}{\lVert {\Delta} \rVert}\right) + 1 & \text{for}\: \lVert {\Delta} \rVert > \epsilon \wedge \lVert {r_{{t}} \rVert} > \epsilon \\\frac{1}{\lVert {u_{{t}} \rVert}^{3} \mu} \left(1.0 \lVert {u_{{t}} \rVert} \mu u_{{t1}}^{2} - \frac{\mu u_{{t2}}^{2}}{2} \lVert {u_{{t}} \rVert} \theta\left(\lambda^{{\prime\prime}}_{1}\right) - \frac{\mu u_{{t2}}^{2}}{2} \lVert {u_{{t}} \rVert} \theta\left(- 2 \lVert {u_{{t}} \rVert} \mu + \lambda^{{\prime\prime}}_{1}\right) + 1.0 \lVert {u_{{t}} \rVert} \mu u_{{t2}}^{2} - \frac{u_{{t1}}^{2}}{2} \max\left(0, \lambda^{{\prime\prime}}_{1}\right) + \frac{u_{{t1}}^{2}}{2} \max\left(0, - 2 \lVert {u_{{t}} \rVert} \mu + \lambda^{{\prime\prime}}_{1}\right)\right) & \text{for}\: \lVert {r_{{t}} \rVert} \leq \epsilon \wedge \lVert {\Delta} \rVert > \epsilon \wedge \lVert {u_{{t}} \rVert} > \epsilon \end{cases}
\end{equation}

\subsection{The Fischer Burmeister function}

\section{Conclusion}

Symbolic differentiation -> inefficient code

A better approach Symbolic code generation for the function, Automatic
differentiation of the smooth part + user provided values on nonsmooth part.



\bibliographystyle{plainnat}
\bibliography{biblio}

\end{document}
\endinput

%%% Local Variables: 
%%% mode: latex
%%% TeX-master: t
%%% TeX-engine: default-shell-escape 
%%% End: 
